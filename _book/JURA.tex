\documentclass[]{book}
\usepackage{lmodern}
\usepackage{amssymb,amsmath}
\usepackage{ifxetex,ifluatex}
\usepackage{fixltx2e} % provides \textsubscript
\ifnum 0\ifxetex 1\fi\ifluatex 1\fi=0 % if pdftex
  \usepackage[T1]{fontenc}
  \usepackage[utf8]{inputenc}
\else % if luatex or xelatex
  \ifxetex
    \usepackage{mathspec}
  \else
    \usepackage{fontspec}
  \fi
  \defaultfontfeatures{Ligatures=TeX,Scale=MatchLowercase}
\fi
% use upquote if available, for straight quotes in verbatim environments
\IfFileExists{upquote.sty}{\usepackage{upquote}}{}
% use microtype if available
\IfFileExists{microtype.sty}{%
\usepackage{microtype}
\UseMicrotypeSet[protrusion]{basicmath} % disable protrusion for tt fonts
}{}
\usepackage[margin=1in]{geometry}
\usepackage{hyperref}
\hypersetup{unicode=true,
            pdftitle={Erhvervs- og finansjura},
            pdfauthor={Sonny Kristoffersen Advokat, Partner, lektor og Ph.D.},
            pdfborder={0 0 0},
            breaklinks=true}
\urlstyle{same}  % don't use monospace font for urls
\usepackage{natbib}
\bibliographystyle{apalike}
\usepackage{longtable,booktabs}
\usepackage{graphicx,grffile}
\makeatletter
\def\maxwidth{\ifdim\Gin@nat@width>\linewidth\linewidth\else\Gin@nat@width\fi}
\def\maxheight{\ifdim\Gin@nat@height>\textheight\textheight\else\Gin@nat@height\fi}
\makeatother
% Scale images if necessary, so that they will not overflow the page
% margins by default, and it is still possible to overwrite the defaults
% using explicit options in \includegraphics[width, height, ...]{}
\setkeys{Gin}{width=\maxwidth,height=\maxheight,keepaspectratio}
\IfFileExists{parskip.sty}{%
\usepackage{parskip}
}{% else
\setlength{\parindent}{0pt}
\setlength{\parskip}{6pt plus 2pt minus 1pt}
}
\setlength{\emergencystretch}{3em}  % prevent overfull lines
\providecommand{\tightlist}{%
  \setlength{\itemsep}{0pt}\setlength{\parskip}{0pt}}
\setcounter{secnumdepth}{5}
% Redefines (sub)paragraphs to behave more like sections
\ifx\paragraph\undefined\else
\let\oldparagraph\paragraph
\renewcommand{\paragraph}[1]{\oldparagraph{#1}\mbox{}}
\fi
\ifx\subparagraph\undefined\else
\let\oldsubparagraph\subparagraph
\renewcommand{\subparagraph}[1]{\oldsubparagraph{#1}\mbox{}}
\fi

%%% Use protect on footnotes to avoid problems with footnotes in titles
\let\rmarkdownfootnote\footnote%
\def\footnote{\protect\rmarkdownfootnote}

%%% Change title format to be more compact
\usepackage{titling}

% Create subtitle command for use in maketitle
\newcommand{\subtitle}[1]{
  \posttitle{
    \begin{center}\large#1\end{center}
    }
}

\setlength{\droptitle}{-2em}

  \title{Erhvervs- og finansjura}
    \pretitle{\vspace{\droptitle}\centering\huge}
  \posttitle{\par}
    \author{Sonny Kristoffersen Advokat, Partner, lektor og Ph.D.}
    \preauthor{\centering\large\emph}
  \postauthor{\par}
      \predate{\centering\large\emph}
  \postdate{\par}
    \date{2019-03-02}

\usepackage{booktabs}
\usepackage{amsthm}
\makeatletter
\def\thm@space@setup{%
  \thm@preskip=8pt plus 2pt minus 4pt
  \thm@postskip=\thm@preskip
}
\makeatother

\begin{document}
\maketitle

{
\setcounter{tocdepth}{1}
\tableofcontents
}
\hypertarget{section}{%
\chapter*{}\label{section}}
\addcontentsline{toc}{chapter}{}

\hypertarget{rs-online}{%
\chapter{RS Online}\label{rs-online}}

Audiolist.

Video sådan køber du adgang.

Video sådan køber du adgang.

Video sådan logger du ind.

\hypertarget{indledning}{%
\chapter{Indledning}\label{indledning}}

--\textgreater{}
 --\textgreater{}

\textbf{\emph{Noterne er kun til personligt brug. Alle rettigheder forbeholdes. Fotografisk eller anden gengivelse af eller kopiering eller anden udnyttelse, er uden forfatterens skriftlige samtykke forbudt ifølge dansk lov om ophavsret.}}

\hypertarget{retskilder-retssystemet-og-juridisk-metode}{%
\chapter{Retskilder retssystemet og juridisk-metode}\label{retskilder-retssystemet-og-juridisk-metode}}

\hypertarget{grundloven-om-danmarks-forfatning}{%
\section{Grundloven om Danmarks forfatning}\label{grundloven-om-danmarks-forfatning}}

Af de mere end 1.300 love, vi har i Danmark, er der én lov, en retskilde, der er hævet over alle de andre, nemlig grundloven. Grundloven er Danmarks forfatning, dvs. den lov, der beskriver de grundlæggende regler for samfundet. I Danmark fejrer man hvert år grundlovens fødselsdag 5. juni. Men hvordan blev grundloven egentlig indført?
Fra 1660 til 1848 havde Danmark enevælde og i 1700-tallet begyndte modstanden mod enevældet at ulme rundt om i Europa.
Befolkningerne stillede krav om, at folket skulle være med til at bestemme, hvordan deres land skulle styres, og flere steder blev monarkiet væltet og erstattet af en republik.
Efterhånden som den internationale udvikling tog fart, erkendte Kong Christian den 8., at Danmark også måtte have en fri forfatning. Da han døde, gjorde hans søn, Frederik den 7., arbejdet med forfatningen færdig.

Se grundloven her!

\hypertarget{video-om-den-danske-grundlov}{%
\subsubsection{Video om den danske grundlov}\label{video-om-den-danske-grundlov}}

\hypertarget{rettigheder-og-pligter}{%
\section{Rettigheder og pligter}\label{rettigheder-og-pligter}}

Grundloven beskriver bl.a. den enkelte borgers rettigheder og pligter, f.eks. at der er ytringsfrihed, religionsfrihed og værnepligt. I grundloven står der også, at den øverste magt i Danmark skal deles mellem den lovgivende, den udøvende og den dømmende magt.
I grundloven kan du læse om magtens fordeling i samfundet. Om Folketinget som den demokratisk valgte forsamling, der vedtager love, der gælder for os alle. Om regeringen, der skal sikre, at lovene bliver overholdt af os borgere og af de myndigheder, der skal sørge for, at vi for eksempel har gode skoler, sygehuse og biblioteker. Om domstolene, der er uafhængige af regering og Folketing, fordi de skal dømme i konflikter mellem borgerne indbyrdes og mellem myndigheder og borgere.

Grundloven handler også om de rettigheder, du har som borger. Vi kalder dem friheds- eller menneskerettigheder.
Den ene slags frihedsrettigheder er ytringsfriheden, retten til at forsamles og demonstrere for dine synspunkter og retten til at oprette foreninger og til at være medlem af en forening. Grundloven sikrer også, at du har ret til at være medlem af et politisk parti og være politisk aktiv -- også selv om det går imod regeringens eller flertallets synspunkter. Disse rettigheder skal sikre, at demokratiet kan fungere. Grundlovens regler om folkeafstemninger og valg til Folketinget ville for eksempel ikke være meget værd, hvis vi ikke havde ret til at diskutere politiske spørgsmål og sige vores mening.
Den anden slags frihedsrettigheder er reglerne om den personlige frihed og om ejendomsretten og boligens ukrænkelighed. Disse regler skal først og fremmest beskytte borgerne mod overgreb fra statsmagten. Hvis du bliver anholdt af politiet, har du for eksempel krav på, at en dommer tager stilling til din sag inden 24 timer. Hvis myndighederne vil undersøge din bolig, dine private papirer eller din pc, skal de som hovedregel have en dommers tilladelse først. -- Og hvis myndighederne vil tage dit hus for at rive det ned, fordi der skal bygges en motorvej eller en jernbane hen over grunden, ja så skal du have en erstatning, der svarer til husets og grundens værdi. Grundloven sætter på den måde grænser for, hvordan staten kan blande sig i vores privatliv.
Grundloven skal sikre stabile rammer om det politiske liv og de politiske kampe om magten. Og grundloven skal sikre, at borgernes rettigheder ikke krænkes. Begge dele sikres ved, at grundloven er mere vanskelig at ændre end andre love. Den danske grundlov er kun blevet ændret få gange, siden den blev vedtaget for mere end 160 år siden. Og sproget i mange af paragrafferne er ikke blevet moderniseret siden. Derfor er der i dette hæfte nogle forklarende kommentarer til de enkelte paragraffer.

\hypertarget{magtadskillelseslren-i-grundlovens-3}{%
\section{Magtadskillelseslæren i grundlovens § 3}\label{magtadskillelseslren-i-grundlovens-3}}

Grundlovens § 3 har følgende ordlyd: ''Den lovgivende magt er hos kongen og Folketinget i forening. Den udøvende magt er hos kongen. Den dømmende magt er hos domstolene''.

\begin{itemize}
\tightlist
\item
  Lovgivende magt: Folketinget
\item
  Dømmende magt: Domstolene
\item
  Udøvende magt: Regering/ministerier, forvaltningen, politiet m.fl.
\item
  Gensidig kontrol
\item
  Magtbalance
\end{itemize}

Bestemmelsen handler om magtens tredeling i den lovgivende, den udøvende og den dømmende magt. Magten er delt mellem forskellige myndigheder (Folketing, regering og domstole) for at undgå, at al magt samles hos én myndighed. Det ville kunne føre til magtmisbrug.
Ifølge grundloven har dronningen og Folketinget i fællesskab magten til at lovgive. Men helt sådan er det ikke i virkeligheden. I praksis er det nemlig regeringen og Folketinget, som bestemmer, hvordan lovene skal se ud. Dronningen skriver dem bare under. Dronningen skal føre lovene ud i livet -- hun har den udøvende magt. I dag betyder det blot, at hun rent formelt udnævner ministrene i en regering. Derefter er det i praksis ministrene og deres ministerier, der sørger for, at lovene bliver overholdt.
Dronningen har ingen indflydelse på, hvem der skal være ministre. Det bestemmer statsministeren. Hun har heller ingen indflydelse på, hvilke partier der skal danne regering. Det handler grundlovens §§ 12-15 bl.a. om.
Domstolene har magten til at dømme. De afgør, om folk har overtrådt landets love og skal straffes. Og de tager stilling i sager, hvor borgere har indbyrdes konflikter. Domstolene afgør også, om ministerier og kommuner har overtrådt lovene, og om lovene overholder grundloven.
I 1999 fastslog Højesteret, jf. U 1999.841 H, at den såkaldte Tvindlov var i strid med grundlovens § 3. Tvindlovens bestemmelse om, at en række Tvindskoler ikke længere skulle have penge fra det offentlige, var derfor ugyldig.\\
 

\hypertarget{folketinget-som-den-lovgivende-magt}{%
\section{Folketinget som den lovgivende magt}\label{folketinget-som-den-lovgivende-magt}}

Folketinget er Danmarks parlament. Her vedtages al lovgivning i Danmark. Folketingets grundlæggende opgaver og nogle af arbejdsformerne er beskrevet i grundloven. Andre metoder er praksisser, der har udviklet sig gennem de snart 170 år, Folketinget har eksisteret.
Folketinget er den lovgivende magt. Folketinget og regeringen er de eneste, der kan fremsætte lovforslag, dvs. komme med forslag til nye love og lovændringer. Den lovgivende magt er Folketinget og regering, men det er kun Folketinget, der kan vedtage lovforslag.
Grundloven beskriver fordelingen af magten mellem Folketinget (lovgivende), regeringen (udøvende og lovgivende) og domstolene (dømmende), også kaldet magtens tredeling. Magten i samfundet er delt i 3 for at forhindre, at der sker magtmisbrug.

\hypertarget{parlamentarisk-kontrol-og-lovgivning-i-folketinget}{%
\subsection{Parlamentarisk kontrol og lovgivning i Folketinget}\label{parlamentarisk-kontrol-og-lovgivning-i-folketinget}}

Folketinget har 3 hovedopgaver:
* at behandle lovforslag og vedtage landets love
* at behandle og vedtage statens årlige budget, finansloven
* at føre kontrol med regeringens magtudøvelse

For at kunne løfte de opgaver kræves det, at folketingsmedlemmerne ved, hvordan det politiske arbejde skal foregå. Reglerne for, hvordan folketingsmedlemmerne skal samarbejde, og hvordan lovgivningsprocessen er, står beskrevet i Folketingets forretningsorden. Reglerne kan ændre sig over tid, fordi samfundet ændrer sig, men de fleste af reglerne har mange år på bagen.

\hypertarget{lovforslag}{%
\subsection{Lovforslag}\label{lovforslag}}

Love regulerer, hvordan vi skal leve sammen i Danmark, hvad vi skal drive i fællesskab -- f.eks. folkeskolen og sygehusene -- og hvordan det fælles skal være indrettet. Lovgivning handler om, hvad man skal som borger, og hvad man ikke må, herunder hvad der er strafbart. Lovteksterne skal derfor skrives så præcist, at borgerne ikke er i tvivl om de fælles spilleregler i samfundet.

Nye lovforslag kan fremsættes af regeringen og folketingsmedlemmerne (folketingsbeslutninger).
Regeringen fremsætter de fleste lovforslag

Lovforslag tager lang tid at skrive, og det kræver stor juridisk indsigt. Derfor er det regeringen, der fremsætter de fleste lovforslag. For regeringen har mange embedsmænd i ministerierne til at hjælpe sig, mens partierne uden for regeringen, oppositionen, har færre til at hjælpe sig -- de har kun medarbejdere i deres gruppesekretariater og i Folketingets Administration.

Ideer til nye love kommer primært fra regeringen, men kan også komme fra:
* folketingsmedlemmerne
* interesseorganisationer, erhvervslivet, foreninger m.v.
* sager i medierne
* borgere, der henvender sig til et folketingsmedlem eller et parti med en sag\\
Inden et lovforslag fremsættes i Folketingssalen, har det som regel været igennem en længere process i ministerierne.

\hypertarget{folketingets-kontrol-med-regeringen}{%
\section{Folketingets kontrol med regeringen}\label{folketingets-kontrol-med-regeringen}}

Ud over at lovgive har Folketinget en anden og lige så vigtig rolle i demokratiet. Det er at kontrollere, om regeringen -- den udøvende magt ‒ fører Folketingets love ud i livet, som de var tænkt, og om der sker magtmisbrug. Det kalder man parlamentarisk kontrol.

\hypertarget{den-kritiske-opposition}{%
\subsection{Den kritiske opposition}\label{den-kritiske-opposition}}

Parlamentarisk kontrol med regeringen er en meget vigtig opgave i Folketinget, hvor Folketinget kontrollerer, hvordan regeringen fører lovene ud i livet, og om regeringen fører en politik, der i hovedtræk bliver støttet af et flertal i Folketinget.
Søgelyset rettes ofte mod ministrene i regeringen. For det meste går det stille af, men situationen kan også spidse til, så regering og Folketing kommer til at stå stejlt over for hinanden.
I praksis er det oppositionen -- de partier i Folketinget, som er imod regeringens politik -- der udfører den parlamentariske kontrol. Det er dem, der har den største interesse i at være kritiske over for regeringens arbejde og afsløre, om der sker magtmisbrug.
Oppositionen undersøger bl.a., om regeringen:
* virkeliggør lovens indhold og gør det på den måde, som regeringen har lovet
* overholder statsbudgettet\\
Folketingsmedlemmerne fra oppositionspartierne finder den parlamentariske kontrol af regeringens magtudøvelse vigtig. Oppositionen har en politisk fordel i at finde fejl og mangler ved en regering, den ikke er enig med, og det giver oppositionspartierne mulighed for at forklare vælgerne, hvordan deres politik er anderledes end regeringens.

\hypertarget{udvalgsarbejdet-i-folketinget}{%
\subsection{Udvalgsarbejdet i Folketinget}\label{udvalgsarbejdet-i-folketinget}}

Arbejdet i Folketinget foregår både i Folketingssalen og i Folketingets udvalg. Udvalgene arbejder med hvert deres fagområde og behandler både lovgivning og den brede kontrol med regeringens arbejde.

\hypertarget{regeringen-den-udvende-magt}{%
\section{Regeringen den udøvende magt}\label{regeringen-den-udvende-magt}}

Statsministeren er regeringens chef og den der bestemmer, hvem der skal være minister i regeringen. Regeringen laver landets love, mens Folketinget vedtager dem. Ministrene er som regel også medlemmer af Folketinget, men det er ikke et krav.

\hypertarget{sadan-dannes-en-regering}{%
\subsection{Sådan dannes en regering}\label{sadan-dannes-en-regering}}

Grundloven giver dronningen magt til at udpege statsministeren og de øvrige ministre. Men i praksis er det et flertal i Folketinget, der er afgørende for, hvem der skal være statsminister. Statsministeren sætter så sit ministerhold og danner sin regering.

\hypertarget{statsministeren-vlges-og-nedstter-sin-regering}{%
\subsection{Statsministeren vælges og nedsætter sin regering}\label{statsministeren-vlges-og-nedstter-sin-regering}}

Folketingsvalg skal afholdes mindst én gang hvert 4. år. Det står i grundloven. Statsministeren kan dog til enhver tid udskrive folketingsvalg, så der kan sagtens gå mindre end 4 år mellem hvert valg.

Når valget er slut og mandaterne fordelt, er det nye Folketing fundet. Så skal man i gang med at finde ud af, hvem der skal være statsminister og danne regering.
\#\#\# Negativ parlamentarisme
Den siddende statsminister fortsætter, hvis der ikke er et flertal imod ham eller hende - dvs. 90 eller flere ud af de 179 folketingsmedlemmer. Det kaldes negativ parlamentarisme.
Taber regeringen valget, skal der findes en ny statsminister og en ny regering.
\#\#\# Dronningen udnævner formelt den nye regering
Den siddende statsminister skal, i samarbejde med dronningen, finde ud af, hvilken statsministerkandidat der har den bredeste opbakning hos det nye folketingsmedlemmer. Det kaldes en dronningerunde.
Når statsministeren har fundet sine ministre, udnævner dronningen formelt regeringen på Amalienborg. Regeringen kommer bagefter ud på slotspladsen. Her præsenterer statsministeren sin nye regering for danskerne.

\hypertarget{sadan-arbejder-regeringen}{%
\subsection{Sådan arbejder regeringen}\label{sadan-arbejder-regeringen}}

Regeringens ministre leder via deres ministerier landet efter de love, Folketinget har vedtaget. Regeringen har stor indflydelse på de love, der vedtages i Folketinget, da det er regeringen, der kommer med langt de fleste forslag til nye love.

\hypertarget{regeringen-fremstter-lovforslagene}{%
\subsection{Regeringen fremsætter lovforslagene}\label{regeringen-fremstter-lovforslagene}}

Den udøvende magt ligger hos regeringen. Det står i grundloven. Regeringen har ret og pligt til at gennemføre de regler og love, som Folketinget vedtager, så lovene bliver til virkelighed i samfundet.
Lovgivningsarbejdet har regeringen også stor indflydelse på. Det er regeringen, der forbereder og foreslår langt de fleste nye love og ændringer i eksisterende love. Regeringen er ansvarlig for hele processen med forberedelse af lovforslagene inden de når til Folketinget, herunder at lægge de overordnede planer, drive de politiske forhandlinger, skrive lovudkast og sende det i høring.
Når regeringen foreslår en lov, kaldes det, at den fremsætter et lovforslag. Både regeringen og medlemmerne af Folketinget kan fremsætte lovforslag -- men de fleste kommer fra regeringen.

\hypertarget{lovkataloget-viser-regeringens-mal}{%
\subsection{Lovkataloget viser regeringens mål}\label{lovkataloget-viser-regeringens-mal}}

Når en ny regering tiltræder, vil de partier, der indgår i regeringen, sammen skrive et regeringsgrundlag. Et regeringsgrundlag er en slags politisk programerklæring, som signalerer, hvilken politik regeringen gerne vil gennemføre i den kommende regeringsperiode.

Lovkataloget er et redskab for regeringen til at omsætte den overordnede politik til konkrete initiativer. Regeringen forsøger at gøre de politiske mål til virkelighed via nye love. De lovforslag, regeringen regner med at kunne få flertal for, bliver præsenteret i regeringens lovkatalog eller lovprogram, som udarbejdes og præsenteres for et folketingsår ad gangen.
Lovkataloget kan findes på Statsministeriets hjemmeside og giver et pejlemærke for årets politiske beslutninger.

Den formelle baggrund for lovkataloget er, at der i grundlovens § 38 står beskrevet, at statsministeren skal lave en åbningsredegørelse, når et nyt folketingsår starter. Åbningsredegørelsen består både af en mundtlig og en skriftlig del. Den mundtlige del er åbningstalen, som statsministeren holder på åbningsdagen, mens den skriftlige del er lovkataloget.

\hypertarget{statsministeren}{%
\subsection{Statsministeren}\label{statsministeren}}

Statsministen er regeringens chef. Det er statsministeren, der bestemmer, hvem der skal være ministre, og hvilke ministerier der skal være.

\hypertarget{statsministeren-har-stor-magt}{%
\subsection{Statsministeren har stor magt}\label{statsministeren-har-stor-magt}}

Statsministerposten er landets højeste ministerpost. En statsminister har en særlig stor magt og et særlig stort ansvar. Det er f.eks. kun statsministeren, der kan bestemme:

\begin{itemize}
\tightlist
\item
  hvem der skal være minister
\item
  hvilke ministre der eventuelt skal afskediges/udskiftes
\item
  hvornår der skal være folketingsvalg inden for den 4-årige regeringsperiode
\end{itemize}

Statsministeren fører tilsyn med sine ministre og fordeler opgaver og fagområder imellem dem.

\hypertarget{ministrene}{%
\subsection{Ministrene}\label{ministrene}}

Regeringen, og dermed ministrene, har den udøvende magt i Danmark. Det vil sige, at ministrene har det overordnende ansvar for, at de love, som Folketinget vedtager, føres ud i livet. Ministrene har meget magt og ansvar, og en af deres vigtigste opgaver er at foreslå ny lovgivning.

Regeringens ministre har hver sit fagområde: Kulturministeren har ansvar for kulturområdet, skatteministeren for skatteområdet osv.
De fleste ministerområder ligger nogenlunde fast -- f.eks. er der i praksis altid et Justitsministerium, et Finansministerium osv. Men statsministeren kan ændre ministrenes fagområder eller oprette nye ministerier, hvis statsministeren ønsker at begrænse, fremhæve eller styrke bestemte fagområder. F.eks. blev der efter valget i november 2007 som noget nyt udnævnt en minister for klima og energi.

\hypertarget{ministrenes-opgaver}{%
\subsection{Ministrenes opgaver}\label{ministrenes-opgaver}}

At foreslå ny lovgivning er en af ministrenes fornemste opgaver. Lovforslagene bliver forberedt i ministerierne, som har mange medarbejdere til bl.a. at skrive lovforslag.

En ministers arbejde består bl.a. i at:

\begin{itemize}
\tightlist
\item
  forhandle indholdet af nye love på plads
\item
  svare på spørgsmål fra Folketinget, f.eks. fra Folketingets udvalg
\item
  svare på henvendelser fra borgere og organisationer m.fl.
\item
  informere offentligheden om ministeriets arbejde, bl.a. i form af hjemmesider, interviews og pressemeddelelser
\item
  træffe beslutninger i ministeriet og sørge for, at ministeriet arbejder effektivt
\item
  samarbejde med internationale parter, herunder EU
\item
  deltage i ugentlige ministermøder med resten af regeringens ministre
\item
  deltage i statsrådsmøder og regeringens udvalg
\item
  deltage i regeringsseminarer
\end{itemize}

\hypertarget{domstolene}{%
\section{Domstolene}\label{domstolene}}

Uafhængige domstole er en grundlæggende del af magtens tredeling i et demokrati. Sådan er det også i Danmark. Grundloven siger nemlig, at domstolene alene har den dømmende magt.

\hypertarget{domstolenes-opgaver}{%
\subsection{Domstolenes opgaver}\label{domstolenes-opgaver}}

Domstolene er den dømmende magt i Danmark. De afgør, om personer har overtrådt landets love, og afgør uoverensstemmelser mellem to parter i civile sager.

Alle har ret til en retfærdig rettergang. Det står i den europæiske menneskerettighedskonvention. Det betyder bl.a., at en retssag skal afgøres inden for en rimelig tidsperiode og ved en domstol, der er uafhængig og upartisk.\\
Domstolenes uafhængighed er bestemt i den danske grundlov. Grundloven deler nemlig magten i tre for at forhindre magtmisbrug, jf. grundlovens § 3:

\begin{itemize}
\tightlist
\item
  den lovgivende magt (Folketinget og regeringen)
\item
  den udøvende magt (regeringen)
\item
  den dømmende magt (domstolene)
\end{itemize}

Det er Folketinget, der vedtager Danmarks love. Regeringen regerer ud fra lovene. Men hverken Folketinget eller regeringen kan dømme på baggrund af de vedtagne love. Kun domstolene kan afgøre, hvordan lovene skal fortolkes, og dømme ud fra dem. Dommerne må kun rette sig efter, hvad der står i loven og det forarbejde, der ligger til grund for den. De må ikke lade sig påvirke af Folketinget, regeringen, pressen eller andre, når de dømmer i en sag.

\hypertarget{byret-landsret-og-hjesteret}{%
\subsection{Byret, landsret og Højesteret}\label{byret-landsret-og-hjesteret}}

Domstolene i Danmark har 3 instanser: byret, landsret og Højesteret. Alle retssager begynder som udgangspunkt i en byret. Almindelige borgere kan være med til at dømme i straffesager som domsmænd eller nævninge, afhængigt af sagens karakter.
De almindelige domstole behandler civile sager og straffesager.
* Civile sager er sager, som anlægges ved domstolene for at få afgjort en uenighed mellem 2 parter. Som eksempler på civile sager kan nævnes sager om mangler ved fast ejendom, opsigelse af en arbejdstager, boligretssager, ægteskabssager, faderskabssager og sager om adoption.
* Straffesager er først og fremmest sager, hvor retten skal træffe beslutning, om en person skal straffes for en overtrædelse af loven. Afgørelser, der træffes i forbindelse med politiets efterforskning, er også straffesager. Det kan f.eks. være afgørelser om varetægtsfængsling, beslaglæggelse og ransagning.
* Byretterne behandler ligeledes skiftesager, foged- og auktionssager.
* Tinglysning af dokumenter foregår ved Tinglysningsretten.

Alle sager kan som udgangspunkt behandles ved 2 retsinstanser, f.eks. ved byret og landsret. Visse mindre sager kan dog normalt kun behandles ved én instans, byretterne, uden mulighed for appel til landsretten.

Byretten
Danmark har 24 byretter. De er fordelt over hele landet. Byretten behandler som nævnt bl.a.:
* civile sager
* straffesager
* tinglysningssager
* skiftesager

\hypertarget{hjesteret}{%
\subsection{Højesteret}\label{hjesteret}}

Danmark har én højesteret. Den ligger ved Christiansborg Slot i København. Højesteret er den øverste domstol i Danmark. Her afgøres f.eks. sager, der har betydning for, hvordan lignende sager skal afgøres, eller sager, der har særlig samfundsmæssig interesse.
Højesteret er en appeldomstol, som behandler domme og kendelser, der er afsagt af Østre Landsret, Vestre Landsret eller Sø- og Handelsretten. Man kan altså ikke anlægge sag direkte ved Højesteret.
Højesteret behandler både civile sager og straffesager og fungerer som tredje instans i skifte-, foged- og tinglysningssager. Ved Højesteret behandles også sager, som Procesbevillingsnævnet har tilladt indbragt for retten. Det drejer sig om sager af særlig principiel betydning, f.eks. sager, som kan have betydning for afgørelse af en række andre sager, eller om sager af særlig samfundsmæssig interesse.
I straffesager kan Højesteret ikke tage stilling til skyldsspørgsmålet.
Særlige forvaltnings- eller forfatningsdomstole anvender man i mange andre EU-lande. Det gør man ikke i Danmark. Her er det de almindelige domstole -- i praksis Højesteret -- der undersøger, om de afgørelser, myndighederne træffer, er lovlige, eller om en lov er i strid med grundloven.

\hypertarget{landsretterne}{%
\subsection{Landsretterne}\label{landsretterne}}

I Danmark findes der 2 landsretter: Vestre Landsret i Viborg og Østre Landsret i København. Landsretten behandler primært appelsager fra byretten.
Hvis byretten f.eks. mener, at en sag er af principiel karakter, kan den også henvise sagen direkte til behandling i landsretten.

Østre Landsret i Bredgade 55, København K

Ved landsretterne er ansat ca. 100 landsdommere. Hver landsret ledes af en præsident.
Vestre Landsret ligger i Viborg og behandler sager fra Jylland.
Østre Landsret ligger i København og behandler sager fra resten af landet, Færøerne og Grønland.
Landsretterne er delt i afdelinger. Til hver afdeling hører tre landsdommere, som i fællesskab afgør alle afdelingens sager, både straffesager og civile sager. I nogle straffesager medvirker nævninger, i andre domsmænd. I enkelte sager deltager personer, der er særligt sagkyndige på et specielt område.
Vestre og Østre Landsret er appelinstanser for byretterne.

\hypertarget{s--og-handelsretten}{%
\subsection{Sø- og Handelsretten}\label{s--og-handelsretten}}

Sø- og Handelsretten ligger i København. I Sø- og Handelsretten er ansat en præsident, 2 vicepræsidenter, 2 dommere og et antal sagkyndige medlemmer. De sagkyndige medlemmer har særlig kendskab til sø- og handelsforhold.
Sø- og Handelsretten behandler bl.a. sager, hvor kendskab til sø- og handelsforhold er af væsentlig betydning.
Sø- og Handelsretten behandler også sager om konkurs, rekonstruktion og gældssanering m.v. fra hele Storkøbenhavn.

\hypertarget{den-srlige-klageret}{%
\subsection{Den Særlige Klageret}\label{den-srlige-klageret}}

Den Særlige Klageret ligger i København samme sted som Højesteret. Klageretten er sammensat af en højesteretsdommer, en landsdommer, en byretsdommer, en advokat og en universitetslærer i retsvidenskab eller anden jurist med særlig videnskabelig uddannelse.
Den Særlige Klageret træffer afgørelser i disciplinære sager vedrørende dommere og andet juridisk personale ansat ved domstolene, herunder også ansatte på Færøerne og i Grønland.

Klageretten behandler derudover sager om genoptagelse af straffesager og udelukkelse af forsvarere fra straffesager.

\hypertarget{byretterne}{%
\subsection{Byretterne}\label{byretterne}}

Danmark er inddelt i 24 såkaldte byretskredse.
Byretterne ledes af en byretspræsident.
Byretterne behandler civile sager, straffesager, notarialforretninger og skiftesager.
Alle sager begynder som udgangspunkt i byretten.
Byretterne kan i en række tilfælde henvise en civil sag til behandling ved landsretten. Det gælder f.eks., hvis sagen er af principiel betydning, eller sagen kan få væsentlig betydning for andre end parterne.

(Kildefigur; Trojka)

\hypertarget{video-hvordan-fungerer-byretten}{%
\subsubsection{Video hvordan fungerer byretten?}\label{video-hvordan-fungerer-byretten}}

\hypertarget{video-en-sag-kommer-pa-tvrs}{%
\subsubsection{Video en sag kommer på tværs}\label{video-en-sag-kommer-pa-tvrs}}

Saglig kompetence: Hvilken ret/domstol skal behandle sagen?

Stedlig kompetence: Værneting -- Hvor i landet skal sagen anlægges?
Hvis der ikke er lavet en værnetingsaftale mellem de stridende parter, skal en retssag som hovedregel anlægges ved sagsøgtes hjemting (bopæl/kendt opholdssted) Værneting er således et spørgsmålet om, hvor en retssag skal anlægges og føres. Ved hvilken domstol en sag skal anlægges, hvis der opstår uenighed mellem aftaleparter med bopæl i hver sit EU-land, afhænger af, om der er indgået en værnetingsaftale og af EU-Domsforordningen.
Supplerende værneting, f.eks.:
* Virksomhedsværneting
* Ejendomsværneting
* Opfyldelsesværneting
* Forbrugerværneting
* Deliktsværneting

(Kildefigur; Trojka)

\hypertarget{processuelle-grundbegreber}{%
\subsection{Processuelle grundbegreber}\label{processuelle-grundbegreber}}

\begin{itemize}
\tightlist
\item
  Forhandlingsprincippet: Sagsøger og sagsøgte har selv ansvaret for sagens bevisførelse. Retten kan opfordre parterne til at føre et bestemt bevis, men parterne er ikke forpligtet til at følge rettens opfordring.
\item
  Bevisumiddelbarhedsprincippet: Beviser skal føres umiddelbart foran dommeren.
\item
  Den frie bevisbedømmelse: Retten har frihed til på objektivt grundlag, at vurdere og afgøre, hvad der findes bevist under sagen, og hvilke beviser der vægter tungere end andre.
\item
  Bevisbyrde: Den som påstår noget under en retssag, skal bevise at han eller hun har ret og har dermed bevisbyrden for sin påstand (kaldet en ''ligefrem bevisbyrde'' som er hovedreglen i dansk ret).
\end{itemize}

\hypertarget{video-retssikkerhed.-hvordan-kan-domstolene-hjlpe-mig}{%
\subsubsection{Video: Retssikkerhed. Hvordan kan domstolene hjælpe mig?}\label{video-retssikkerhed.-hvordan-kan-domstolene-hjlpe-mig}}

\hypertarget{civilretssagens-forlb}{%
\subsection{Civilretssagens forløb}\label{civilretssagens-forlb}}

Parterne: Sagsøger og sagsøgte
Sagens forberedelse:
Stævning og svarskrift
* Evt. forberedende retsmøde
* Evt. syn og skøn
* Replik og duplik
* Hovedforhandling

(Kildefigur; Trojka)
Småsagsprocessen:
* Sager under 50.000 kr.
* Formål: Enklere, hurtigere og billigere

\hypertarget{udenretlig-tvistlsningsorganer}{%
\subsection{Udenretlig tvistløsningsorganer}\label{udenretlig-tvistlsningsorganer}}

Klage- og ankenævn som der har betydning for den finansielle branche:
* fx Forbrugerklagenævn, Det Finansielle Ankenævn (Sammenlægning pr. 1. februar 2019 af Pengeinstitutankenævnet, Realkreditankenævnet og Ankenævnet for investeringsfonde) Ankenævnet for Forsikring, Ankenævnet for Finansieringsselskaber og Klagenævnet for Ejendomsformidling
* Sagsbehandling efter officialmaksimen, dvs. nævnets sekretariatet skal oplyse klagesagen
* Mediation i det offentlige forbrugerklagenævn
Retsmægling
Voldgift
* Voldgiftsloven -- voldgiftsklausul i aftale
* Ofte hurtigere sagsbehandling end alm. domstole
* Dommere med særligt fagkundskab
* Sagen er ikke offentlig
* Ikke mulighed for anke til højere instans

\hypertarget{tinglysningsretten}{%
\subsection{Tinglysningsretten}\label{tinglysningsretten}}

Tinglysningsretten, som ligger i Hobro, blev etableret den 1. januar 2007 med en præsident som øverste chef.
Fra den 8. september 2009 er håndteringen af hele tinglysningsområdet samlet i Tinglysningsretten. Byretterne tinglyser altså ikke længere dokumenter.
Tinglysning er nu en digital proces, der foregår fra hjemmesiden www.tinglysning.dk.
Information og vejledning kan findes på www.tinglysningsretten.dk

\hypertarget{procesbevillingsnvnet}{%
\subsection{Procesbevillingsnævnet}\label{procesbevillingsnvnet}}

Procesbevillingsnævnet blev oprettet pr. 1. januar 1996 med det formål at behandle ansøgninger om 2. og 3. instansbevillinger i civile sager og straffesager. Siden den 1. januar 2007 har Procesbevillingsnævnet endvidere haft til opgave at behandle klager over Civilstyrelsens afslag på ansøgninger om fri proces.
Appeltilladelse
En appeltilladelse indebærer, at en sag, der ikke ellers ville kunne ankes eller kæres, kan indbringes for en højere retsinstans. Appeltilladelse forudsætter, at sagen rejser spørgsmål af principiel karakter, eller at særlige grunde taler for tilladelse.
Fri proces
Fri proces indebærer i grundtræk, at den pågældende får en advokat beskikket til at føre sagen, og at statskassen betaler sagens omkostninger, herunder retsafgifter, advokatsalær og eventuelle sagsomkostninger, som skal betales til modparten. Fri proces er især forbeholdt personer, som på den ene side ikke selv har økonomisk mulighed for at betale sagsomkostningerne, men som på den anden side har rimelig grund til at føre proces, navnlig fordi der er udsigt til at vinde sagen.
Procesbevillingsnævnets afdeling for appeltilladelser består af en højesteretsdommer (formand), en landsdommer, en byretsdommer, en advokat og et medlem, der repræsenterer retsvidenskaben.
Nævnets afdeling for fri proces består af en landsdommer (afdelingsformand), en byretsdommer og en advokat.
Beslutningerne på nævnsmøderne træffes ved almindelig stemmeflerhed. Der afholdes i almindelighed møde i hver afdeling ugentligt, og sagerne udsendes ca. en uge før mødet. Hastesager kan dog udsendes med kortere varsel. Nævnsmøderne er ikke offentlige, og der er ikke mulighed for at få foretræde for nævnet.
I overensstemmelse med forarbejderne til lovgivningen om Procesbevillingsnævnets virksomhed begrundes nævnets afgørelser alene med en henvisning til indholdet af de bestemmelser, der giver hjemmel for meddelelse af bevilling.
Bevillingsmæssigt og administrativt hører Procesbevillingsnævnet under Domstolsstyrelsen, men Procesbevillingsnævnet er uafhængigt af domstolene og af den offentlige forvaltning. Man kan derfor ikke klage over nævnets afgørelser til justitsministeren eller til Folketingets Ombudsmand.

\hypertarget{rigsretten}{%
\subsection{Rigsretten}\label{rigsretten}}

Rigsretten består af indtil 15 af Højesterets dommere og et tilsvarende antal medlemmer, som udpeges af Folketinget. Folketingsmedlemmer kan ikke udpeges til medlem af Rigsretten.
Medlemmerne af Rigsretten vælges for 6 år. Der er udpeget medlemmer af Rigsretten i 1996.

Rigsretten har til opgave at påkende sager mod ministre vedrørende deres embedsførelse. Det er Folketinget, der beslutter, om der skal rejses tiltale mod en minister.
Folketinget har 5 gange rejst tiltale for Rigsretten. Den seneste sag var mod tidligere justitsminister Erik Ninn-Hansen. Der blev afsagt dom i sagen den 22. juni 1995.

\hypertarget{dommerudnvnelsesradet}{%
\subsection{Dommerudnævnelsesrådet}\label{dommerudnvnelsesradet}}

Dommerudnævnelsesrådet er et uafhængigt råd, der har til opgave at afgive indstillinger til justitsministeren vedrørende besættelse af dommerstillinger.
Domstolsstyrelsen stiller sekretariat til rådighed for Dommerudnævnelses-rådet.
Dommerudnævnelsesrådet blev oprettet ved lov nr. 402 af 26. juni 1998 som et led i domstolsreformen, der bl.a. havde til formål at sikre en bredere rekruttering af dommere og større gennemsigtighed ved besættelse af dommerstillinger. Rådet har været i funktion siden den 1. juli 1999, da loven trådte i kraft.

\hypertarget{domstolsstyrelsen}{%
\subsection{Domstolsstyrelsen}\label{domstolsstyrelsen}}

Domstolsstyrelsen har til opgave at administrere og udvikle Danmarks Domstole.
Domstolsstyrelsen blev oprettet som en ny, selvstændig statsinstitution den 1. juli 1999.
Domstolsstyrelsen ledes af en bestyrelse og en direktør. Domstolsstyrelsen hører ressortmæssigt under Justitsministeriet, men justitsministeren kan ikke bestemme over styrelsen og kan ikke ændre styrelsens afgørelser.
Bestyrelsen er den øverste ledelse og har det overordnede ansvar for Domstolsstyrelsens virksomhed. Den daglige ledelse er lagt i hænderne på direktøren, som er ansat af og kan afskediges af bestyrelsen.
Sammensætningen af Domstolsstyrelsens bestyrelse er fastsat i lov om Domstolsstyrelsen.

\hypertarget{video-historien-om-de-danske-domstole-og-retssystemet}{%
\subsubsection{Video: historien om de danske domstole og retssystemet}\label{video-historien-om-de-danske-domstole-og-retssystemet}}

\hypertarget{eu-domstolen}{%
\subsection{EU-domstolen}\label{eu-domstolen}}

Den Europæiske Unions Domstol (EU-Domstolen) fortolker EU's lovgivning for at sikre, at den anvendes på samme måde i alle EU-lande, og afgør retstvister mellem nationale regeringer og EU's institutioner.

I visse tilfælde kan den også anvendes af enkeltpersoner, virksomheder eller organisationer til at gribe ind over for en EU-institution, hvis de mener, den på den ene eller anden måde har tilsidesat deres rettigheder.

\hypertarget{hvad-laver-eu-domstolen}{%
\subsection{Hvad laver EU-Domstolen?}\label{hvad-laver-eu-domstolen}}

Domstolen træffer afgørelser i de sager, der indbringes for den. De mest almindelige typer sager er:
* Fortolkning af love (præjudicielle afgørelser) - de nationale domstole i EU-landene skal sikre, at EU-lovene anvendes korrekt, men domstolene i de forskellige lande kan fortolke reglerne forskellige. Hvis en national domstol er i tvivl om fortolkningen eller gyldigheden af en EU-retsakt, kan den spørge Domstolen til råds. Den samme mekanisme kan anvendes til at afgøre, om national ret eller praksis er forenelig med EU-retten.
* Håndhævelse af loven (overtrædelsesprocedurer) - denne form for sag føres mod et medlemsland, hvis det ikke overholder EU-reglerne. Den kan indledes af Europa-Kommissionen eller et andet EU-land. Hvis landet findes skyldigt, skal det straks bringe bruddet til ophør, hvis ikke det vil risikere, at der anlægges endnu en sag, som kan medføre bødestraf.
* Ophævelse af EU-retsakt (annullationssøgsmål) -- hvis en EU-retsakt menes at være i strid med EU-traktaterne eller grundlæggende rettigheder, kan Domstolen blive bedt om at ophæve den - af et medlemsland, Rådet, Europa-Kommissionen eller (i visse tilfælde) Europa-Parlamentet.
Privatpersoner kan også bede Domstolen om at ophæve en EU-retsakt, som berører dem direkte.
* Sikring af, at EU træffer foranstaltninger (passivitetssøgsmål) -- Parlamentet, Rådet og Kommissionen skal træffe visse afgørelser i bestemte situationer. Undlader de dette, kan EU's institutioner eller (i visse tilfælde) enkeltpersoner eller virksomheder klage til Domstolen.
* Sanktionering af EU-institutionerne (erstatningssøgsmål) -- enhver person eller virksomhed, hvis interesser er blevet skadet som følge af EU's eller dets personales handlinger eller mangel på handlinger, kan bringe sagen for Domstolen.

\hypertarget{sammenstning}{%
\subsection{Sammensætning}\label{sammenstning}}

EU-Domstolen er inddelt i 2 organer:
* EU-Domstolen -- som tager sig af anmodninger om præjudicielle afgørelser fra nationale domstole, visse annullationssøgsmål og appelsager.\\
* Retten -- som træffer afgørelse i annullationssøgsmål indbragt af enkeltpersoner, virksomheder og i visse tilfælde medlemslande. Det vil i praksis sige, at Retten primært beskæftiger sig med konkurrenceret, statsstøtte, handel, landbrug og varemærker.\\
Hver dommer og generaladvokat udnævnes i fællesskab af medlemslandene for en periode på 6 år, som kan fornyes. I hver domstol vælger dommerne en formand for en periode på 3 år, som kan forlænges.

\hypertarget{hvordan-fungerer-eu-domstolen}{%
\subsection{Hvordan fungerer EU-Domstolen?}\label{hvordan-fungerer-eu-domstolen}}

I Domstolen tildeles hver sag én dommer (refererende dommer) og én generaladvokat. Sagerne behandles i to faser:\\
* Den skriftlige fase\\
+ Parterne afgiver skriftlige erklæringer til Domstolen -- og nationale myndigheder, EU-institutionerne og i visse tilfælde enkeltpersoner kan også fremsætte bemærkninger.\\
+ Alt dette sammenfattes af den refererende dommer og drøftes på Domstolens almindelige møde, som beslutter:\\
+ Hvor mange dommere, der skal behandle sagen: 3, 5 eller 15 dommere (hele Domstolen) afhængig af sagens betydning og kompleksitet. De fleste sager behandles af 5 dommere, og det sker meget sjældent, at en sag høres af hele Domstolen.\\
+ Om det er nødvendigt med en høring (den mundtlige fase), og om der er behov for en officiel udtalelse fra generaladvokaten.\\
* Den mundtlige fase -- en offentlig høring.\\
+ Parternes advokater forelægger deres sag for dommerne og generaladvokaten, som kan stille spørgsmål til dem.\\
+ Hvis Domstolen har besluttet, at der er behov for en udtalelse fra generaladvokaten, afgives denne nogle uger efter høringen.\\
+ Dommerne voterer så og kommer med deres afgørelse.\\
* Rettens procedure er lignende, bortset fra at de fleste sager høres af 3 dommere, og at der ikke er nogen generaladvokat.

\hypertarget{eu-domstolen-og-dig}{%
\subsection{EU-Domstolen og dig}\label{eu-domstolen-og-dig}}

Hvis du -- som privatperson eller virksomhed -- har lidt skade som følge af handlinger eller mangel på handlinger fra EU's institutioners eller ansattes side, kan du indbringe en sag for Domstolen på to måder:\\
* indirekte gennem nationale domstole (som kan beslutte at henvise sagen til EU-Domstolen)\\
* direkte for Retten -- hvis en afgørelse truffet af en EU-institution har berørt dig direkte og individuelt.\\
Hvis du mener, at myndighederne i et medlemsland har overtrådt EU-reglerne, skal du følge den officielle klageprocedure. (link)

\hypertarget{video-eu-domstolen}{%
\subsubsection{Video: EU-domstolen:}\label{video-eu-domstolen}}

Med ikrafttrædelsen af Lissabontraktaten den 1. december 2009 har Den Europæiske Union fået status som juridisk person og har overtaget de beføjelser, som tidligere var tildelt Det Europæiske Fællesskab. Fællesskabsretten er således blevet til unionsretten, som også omfatter alle de bestemmelser, der tidligere er blevet vedtaget i medfør af traktaten om Den Europæiske Union som affattet forud for Lissabontraktaten. I den præsentation, som følger, vil udtrykket fællesskabsretten ikke desto mindre blive anvendt, når der henvises til Domstolens praksis før ikrafttrædelsen af Lissabontraktaten.
Ved siden af Den Europæiske Union fortsætter Det Europæiske Atomenergifællesskab (Euratom) med at eksistere. Eftersom Domstolens beføjelser vedrørende Euratom i princippet er de samme som dem, der udøves inden for rammerne af Den Europæiske Union, og for at gøre præsentationen mere læsevenlig, vil enhver henvisning til unionsretten ligeledes omfatte Euratomretten.

\hypertarget{sammenstning-1}{%
\subsection{Sammensætning}\label{sammenstning-1}}

Domstolen består af 28 dommere og 11 generaladvokater. Dommerne og generaladvokaterne udnævnes for en periode af 6 år af medlemsstaternes regeringer efter fælles overenskomst efter høring af et udvalg, som har til opgave at udtale sig om, hvorvidt de indstillede kandidater er egnede til at varetage de omhandlede opgaver. De kan genudnævnes. Til dommere og generaladvokater ved Domstolen
udnævnes personer, hvis uafhængighed er uomtvistelig. De skal i deres hjemland opfylde betingelserne
for at indtage de højeste dommerembeder eller have faglige kvalifikationer, som er almindeligt anerkendt.
Domstolens dommere vælger af deres midte Domstolens præsident og vicepræsident for et tidsrum af tre år. Begge kan genvælges. Præsidenten forestår Domstolens arbejde og administration og leder retsmøderne og Domstolens voteringer i sager, der er henvist til behandling i et af de største dommerkollegier. Vicepræsidenten bistår præsidenten i udførelsen af dennes opgaver og træder i præsidentens sted, hvis denne har forfald.
Generaladvokaterne bistår Domstolen og er den behjælpelig ved udførelsen af dens opgaver. De har til opgave, fuldstændig upartisk og uafhængigt, offentligt at fremsætte forslag til afgørelse af de sager, som de forelægges.
Justitssekretæren er institutionens generalsekretær og leder dens tjenestegrene under tilsyn af Domstolens præsident.
Domstolen kan sættes af samtlige medlemmer (plenum), i Den Store Afdeling (15 dommere) eller i afdelinger med 3 eller 5 dommere.
Domstolen sættes af samtlige medlemmer i særlige tilfælde, der er opregnet i statutten for Domstolen (bl.a. når den skal afskedige Den Europæiske Ombudsmand eller et medlem af Europa-Kommissionen, som har tilsidesat sine forpligtelser), og når Domstolen finder, at en sag er af særlig vigtighed.
Den sættes i Den Store Afdeling, når en medlemsstat eller en institution, som er part i sagen, anmoder herom samt i særligt omfattende eller betydelige sager.
De øvrige sager behandles i afdelinger med 5 eller 3 dommere. Formændene for afdelinger med 5 dommere vælges for en periode af 3 år, og formændene for afdelinger med 3 dommere for en periode af 1 år.
Beføjelser
For at kunne varetage sit hverv er Domstolen tillagt vide retlige beføjelser, som den udøver ved de præjudicielle forelæggelser og i de forskellige typer af søgsmål.

\hypertarget{de-forskellige-sagstyper}{%
\section{De forskellige sagstyper}\label{de-forskellige-sagstyper}}

\hypertarget{prjudicielle-forelggelser}{%
\subsection{Præjudicielle forelæggelser}\label{prjudicielle-forelggelser}}

Domstolen samarbejder med samtlige retsinstanser i medlemsstaterne, som er de ordinære retter, på unionsrettens område. For at sikre en effektiv og ensartet anvendelse af unionsretten og for at undgå forskelle i fortolkningen heraf kan - og i visse tilfælde skal - de nationale retter forelægge Domstolen præjudicielle spørgsmål vedrørende fortolkningen af unionsretten, f.eks. med henblik på, at den nationale ret kan efterprøve de nationale bestemmelsers overensstemmelse med unionsretten. Den præjudicielle forelæggelse kan også vedrøre spørgsmål om en unionsretsakts lovlighed.
Domstolen besvarer ikke sådanne spørgsmål ved blot at afgive en udtalelse, men i en dom eller en begrundet kendelse. Den forelæggende ret er bundet af Domstolens fortolkning. Domstolens dom binder på samme måde de øvrige nationale domstole, som måtte skulle træffe afgørelse vedrørende et identisk spørgsmål.
Det er ligeledes gennem præjudicielle forelæggelser, at enhver europæisk borger kan få afklaring på spørgsmål om de bestemmelser i unionsretten, der vedrører ham. Selv om et præjudicielt spørgsmål kun kan forelægges af en national domstol, har parterne i den sag, der verserer for den nationale domstol, medlemsstaterne og EU-institutionerne adgang til at deltage i proceduren for Domstolen. En række af unionsrettens hovedprincipper er blevet defineret på baggrund af præjudicielle spørgsmål, der også er blevet forelagt af nationale domstole, som træffer afgørelse i første instans.

\hypertarget{traktatbrudssgsmal}{%
\subsubsection{Traktatbrudssøgsmål}\label{traktatbrudssgsmal}}

Domstolen har herigennem adgang til at kontrollere, om medlemsstaterne overholder de forpligtelser, der påhviler dem i medfør af unionsretten. Forud for sagens anlæg ved Domstolen har Kommissionen gennemført en procedure, hvorunder vedkommende medlemsstat har fået lejlighed til at svare på de klagepunkter, som er rejst imod den. Hvis denne procedure ikke fører til, at medlemsstaten bringer traktatbruddet til ophør, kan der anlægges en traktatbrudssag ved Domstolen.
En sådan sag kan anlægges enten af Kommissionen - hvilket i praksis er det hyppigst forekommende - eller af en anden medlemsstat. Hvis Domstolen fastslår, at der foreligger et traktatbrud, skal medlemsstaten straks bringe det til ophør. Hvis Domstolen, efter at Kommissionen på ny har indbragt sagen for den, fastslår, at den pågældende medlemsstat ikke har efterkommet dens dom, kan Domstolen pålægge medlemsstaten at betale et fast beløb eller en tvangsbøde. Hvis Kommissionen ikke er blevet underrettet om foranstaltninger til gennemførelse af et direktiv, kan Domstolen imidlertid efter anmodning herom fra Kommissionen pålægge medlemsstaten en økonomisk sanktion allerede fra tidspunktet for afsigelsen af den første traktatbrudsdom.
\#\#\#\# Annullationssøgsmål
I en sådan sag nedlægger sagsøgeren påstand om annullation af en retsakt, der er udstedt af Unionens institutioner, organer, kontorer eller agenturer (bl.a. forordninger, direktiver og beslutninger). Domstolen er forbeholdt kompetencen i sager, der anlægges af en medlemsstat mod Europa-Parlamentet og/eller Rådet (med undtagelse af Rådets retsakter, der vedrører statsstøtte, antidumping og gennemførelsesbeføjelser), eller sager, der anlægges af en EU-institution mod en anden institution. Retten er kompetent til at træffe afgørelse, i første instans, i alle andre sager af denne art, herunder navnlig i sager, der er anlagt af private.
\#\#\#\# Passivitetssøgsmål
Domstolen og Retten har herigennem adgang til at kontrollere, om det er lovligt, at en fællesskabsinstitution forholder sig passivt i en given situation. En sådan sag kan imidlertid først anlægges, efter at institutionen er blevet opfordret til at handle. Når det er fastslået, at undladelsen var ulovlig, har den pågældende institution pligt til at træffe egnede foranstaltninger til at bringe passiviteten til ophør. Beføjelsen til at påkende passivitetssøgsmål er opdelt mellem Domstolen og Retten efter de samme kriterier, som gælder ved annullationssøgsmål.

\hypertarget{appelsager}{%
\subsubsection{Appelsager}\label{appelsager}}

Endelig kan domme og kendelser afsagt af Retten appelleres til Domstolen for så vidt angår retsspørgsmål. Såfremt appellen admitteres, og Domstolen giver appellanten medhold i realiteten, ophæver den Rettens afgørelse. Hvis sagen er moden til påkendelse, kan Domstolen selv træffe afgørelse i sagen. Finder Domstolen ikke, at sagen er moden til påkendelse, hjemviser den sagen til Retten. I tilfælde af hjemvisning er Retten bundet af de afgørelser, som er truffet af Domstolen under appelsagen.

\hypertarget{sagsbehandling}{%
\subsection{Sagsbehandling}\label{sagsbehandling}}

Uanset hvilken sagstype der er tale om, omfatter den en skriftlig fase og i givet fald en mundtlig fase, der er offentlig. Der er imidlertid forskel på sagsbehandlingen for så vidt angår præjudicielle forelæggelser og i de øvrige sager(direkte søgsmål og appelsager).

\hypertarget{sagens-anlg-og-den-skriftlige-forhandling}{%
\subsection{Sagens anlæg og den skriftlige forhandling}\label{sagens-anlg-og-den-skriftlige-forhandling}}

\hypertarget{prjudicielle-forelggelser-1}{%
\subsubsection{Præjudicielle forelæggelser}\label{prjudicielle-forelggelser-1}}

Den nationale domstol forelægger Domstolen spørgsmål om fortolkningen eller gyldigheden af en bestemmelse i unionsretten, hvilket sædvanligvis sker i form af en retsafgørelse, alt efter de nationale retsplejeregler. Når anmodningen af Domstolens oversættelsestjeneste er oversat til alle unionssprogene, forkynder justitssekretæren den for parterne i hovedsagen samt for medlemsstaterne og EU-institutionerne. Justitssekretæren lader en meddelelse, der indeholder de pågældende parters navne samt spørgsmålene, offentliggøre i Den Europæiske Unions Tidende. Parterne, medlemsstaterne og institutionerne har herefter en frist på to måneder til at indgive skriftlige indlæg til Domstolen.
Den nationale domstol forelægger Domstolen spørgsmål om fortolkningen eller gyldigheden af en bestemmelse i unionsretten, hvilket sædvanligvis sker i form af en retsafgørelse, alt efter de nationale retsplejeregler. Når anmodningen af Domstolens oversættelsestjeneste er oversat til alle unionssprogene, forkynder justitssekretæren den for parterne i hovedsagen samt for medlemsstaterne og EU-institutionerne. Justitssekretæren lader en meddelelse, der indeholder de pågældende parters navne samt spørgsmålene, offentliggøre i Den Europæiske Unions Tidende. Parterne, medlemsstaterne og institutionerne har herefter en frist på to måneder til at indgive skriftlige indlæg til Domstolen.

\hypertarget{direkte-sgsmal-og-appelsager}{%
\subsubsection{Direkte søgsmål og appelsager}\label{direkte-sgsmal-og-appelsager}}

Sagen anlægges ved Domstolen ved indlevering af en stævning til justitskontoret. Justitssekretæren lader en meddelelse om sagsanlægget offentliggøre i Den Europæiske Unions Tidende, hvori sagsøgerens søgsmålsgrunde og argumenter kort angives. Stævningen forkyndes for de øvrige parter, der har en frist på to måneder til at indgive svarskrift eller replik. I givet fald har sagsøgeren ret til at indgive replik og sagsøgte duplik. Den frist, der er fastsat for fremlæggelse af disse dokumenter skal overholdes.

I begge sagstyper udpeges der af henholdsvis præsidenten og førstegeneraladvokaten en refererende dommer og en generaladvokat, der følger sagen under hele forløbet.

\hypertarget{sagens-oplysning}{%
\subsection{Sagens oplysning}\label{sagens-oplysning}}

Når den skriftlige forhandling er afsluttet, opfordres parterne til inden for en frist på tre uger at tilkendegive, om og hvorfor de ønsker, at der afholdes en mundtlig forhandling. Domstolen beslutter på grundlag af den refererende dommers indstilling, og efter at have hørt generaladvokaten, om der skal ske bevisoptagelse, til hvilket dommerkollegium sagen skal henvises, og om der er grund til at afholde en mundtlig forhandling, for hvilken præsidenten i givet fald fastsætter en dato for.

\hypertarget{offentligt-retsmde-og-generaladvokatens-forslag-til-afgrelse}{%
\subsubsection{Offentligt retsmøde og generaladvokatens forslag til afgørelse}\label{offentligt-retsmde-og-generaladvokatens-forslag-til-afgrelse}}

Besluttes det at afholde en mundtlig forhandling, procederes sagen i et offentligt retsmøde, hvori dommerkollegiet og generaladvokaten deltager. Dommerne og generaladvokaten kan stille parterne de spørgsmål, som de finder hensigtsmæssige. Nogle uger senere fremsætter generaladvokaten i et nyt offentligt retsmøde sit forslag til afgørelse for Domstolen. Generaladvokaten behandler navnlig sagens retlige aspekter i enkeltheder og foreslår i al uafhængighed Domstolen, hvorledes problemet efter hans eller hendes opfattelse skal løses. Hermed er den mundtlige del af sagsbehandlingen afsluttet. Hvis det vurderes, at sagen ikke rejser nye retsspørgsmål, kan Domstolen efter at have hørt generaladvokaten bestemme, at sagen skal påkendes uden forslag til afgørelse.

\hypertarget{dommene}{%
\section{Dommene}\label{dommene}}

Dommerne voterer på grundlag af et domsudkast, som den refererende dommer har udarbejdet. Enhver af dommerne i det pågældende dommerkollegium kan foreslå ændringer. Domstolens afgørelser træffes med stemmeflerhed, og en eventuel dissens anføres ikke. Dommene underskrives kun af de dommere, der har deltaget i den mundtlige votering, hvorunder dommen er blevet vedtaget, med forbehold af reglen om, at den dommer i dommerkollegiet, der har den laveste anciennitet, ikke underskriver dommen, såfremt dette dommerkollegium består af et lige antal dommere. Dommene afsiges i offentligt retsmøde. På Domstolens websted CURIA offentliggøres dommene og generaladvokaternes forslag til afgørelse på afsigelsesdagen, henholdsvis fremsættelsesdagen. I de fleste tilfælde offentliggøres de efterfølgende i Samling af Afgørelser.

\hypertarget{srlige-rettergangsformer}{%
\section{Særlige rettergangsformer}\label{srlige-rettergangsformer}}

\hypertarget{den-forenklede-procedure}{%
\subsection{Den forenklede procedure}\label{den-forenklede-procedure}}

Såfremt et præjudicielt spørgsmål er identisk med et spørgsmål, Domstolen allerede har afgjort, såfremt besvarelsen af et sådant spørgsmål ikke giver anledning til nogen rimelig tvivl, eller såfremt besvarelsen af spørgsmålet klart kan udledes af retspraksis, kan Domstolen efter at have hørt generaladvokaten træffe afgørelse ved begrundet kendelse, i givet fald under henvisning til den tidligere afsagte dom eller den relevante retspraksis.
\#\#\# Den fremskyndede procedure
Den fremskyndede procedure giver Domstolen mulighed for at træffe hurtig afgørelse i uopsættelige sager ved i videst muligt omfang at behandle sagerne hurtigt og tillægge dem absolut prioritet. Efter begæring fra en af parterne kan Domstolens præsident efter forslag fra den refererende dommer og efter at have hørt generaladvokaten og de øvrige parter beslutte at anvende den fremskyndede procedure, når sagens særlige uopsættelighed kræver det. De præjudicielle forelæggelser kan ligeledes underkastes en fremskyndet procedure. I så fald fremsættes begæring herom af den forelæggende ret, som i begæringen skal angive de faktiske omstændigheder, som bevirker, at afgørelsen af det præjudicielle spørgsmål er uopsættelig.

\hypertarget{den-prjudicielle-hasteprocedure-ppu}{%
\subsection{Den præjudicielle hasteprocedure (PPU)}\label{den-prjudicielle-hasteprocedure-ppu}}

Denne procedure gør det muligt for Domstolen inden for en væsentligt afkortet frist at behandle de mest følsomme spørgsmål vedrørende området for frihed, sikkerhed og retfærdighed (politisamarbejde og retligt samarbejde i civile sager og i kriminalsager samt visum, asyl, indvandring og andre politikker i forbindelse med den fri bevægelighed for personer). PPU-sagerne behandles i en afdeling med fem dommere, som er særligt udpeget, og den skriftlige fase gennemføres i praksis elektronisk, og er i væsentligt omfang begrænset, både hvad angår varigheden og antallet af aktører, som har ret til at indgive skriftlige indlæg. De fleste aktører deltager under den mundtlige fase, der er obligatorisk.
\#\#\# Begæring om foreløbige forholdsregler
Der kan ligeledes indgives begæring om udsættelse af gennemførelsen af en retsakt, der er udstedt af en institution, eller om enhver anden foreløbig forholdsregel, som er nødvendig for at forhindre, at en part lider et alvorligt og uopretteligt tab.
\#\#\# Sagsomkostninger
Sagsbehandlingen ved Domstolen er fritaget omkostninger. Til gengæld dækkes omkostningerne til en advokat, der har beskikkelse i en medlemsstat, ved hvilken parterne kan lade sig repræsentere, ikke af Domstolen. Hvis en part er ude af stand til helt eller delvis at betale de omkostninger, der er forbundet med sagen, kan han dog uden at være repræsenteret ved en advokat, ansøge om retshjælp. Ansøgningen skal vedlægges alle de nødvendige oplysninger, som godtgør behovet for retshjælp.
\#\#\# Sprogordning
I direkte søgsmål bliver det sprog, som stævningen er affattet på (hvilket kan være ethvert af Den Europæiske Unions 24 officielle sprog) i princippet sagens processprog, dvs. det sprog, som sagen behandles på. I appelsager anvendes samme processprog som i den appellerede dom eller kendelse fra Retten. For så vidt angår præjudicielle forelæggelser, er processproget det, som den nationale ret har henvendt sig til Domstolen på. Under den mundtlige forhandling i retsmøder er der efter behov simultantolkning til forskellige af Den Europæiske Unions officielle sprog. Dommerne voterer uden brug af tolke på et fælles sprog, der normalt vil være fransk.

\hypertarget{oversigt-over-rettergangsmaden}{%
\subsection{Oversigt over rettergangsmåden}\label{oversigt-over-rettergangsmaden}}

Sagsbehandlingen ved Domstolen
Direkte søgsmål og appelsager Præjudicielle sager
Skriftlig forhandling
Stævning
Stævningen forkyndes for sagsøgte af Justitskontoret
Offentliggørelse af stævningen i Den Europæiske Unions Tidende (serie C)
{[}Foreløbige forholdsregler{]}
{[}Intervention{]}
Svarskrift
{[}Formalitetsindsigelse{]}
{[}Replik og duplik{]} {[}Ansøgning om fri proces{]}
Udpegelse af refererende dommer og generaladvokat Den nationale rets afgørelse om forelæggelse
Oversættelse til Den Europæiske Unions øvrige officielle sprog
Offentliggørelse af de præjudicielle spørgsmål i
Den Europæiske Unions Tidende (C-udgaven)
Forkyndelse for parterne i hovedsagen, medlemsstaterne,
EU-institutionerne, EØS-staterne og EFTA-Tilsynsmyndigheden
Skriftlige indlæg fra parterne, staterne og institutionerne
Den refererende dommer udfærdiger foreløbig rapport
Møde mellem dommere og generaladvokater
Henvisning af sagen til et dommerkollegium
{[}Bevisoptagelse{]}
Mundtlig del
{[}Generaladvokatens forslag til afgørelse{]}
Dommernes votering
DOM
Fakultative led i rettergangsmåden er anført i kantet parentes.
Fed skrift angiver, at der er tale om et offentligt dokument.

\hypertarget{domstolen-i-den-europiske-unions-retsorden}{%
\subsection{Domstolen i Den Europæiske Unions retsorden}\label{domstolen-i-den-europiske-unions-retsorden}}

Med henblik på at opbygge Europa som fællesskab indgik en række stater (i dag i alt 28) indbyrdes en række traktater om oprettelse af De Europæiske Fællesskaber og herefter Den Europæiske Union, udstyret med institutioner, som vedtager retsregler på bestemte områder.
Den Europæiske Unions Domstol udgør Unionens og Det Europæiske Atomenergifællesskabs dømmende myndighed. Den består af Domstolen og Retten, hvis opgave i det væsentlige består i at prøve lovligheden af Fællesskabets retsakter og at sikre en ensartet fortolkning og anvendelse af EU-retten.

Hele vejen gennem sin praksis har Domstolen knæsat pligten for forvaltninger og domstole på nationalt plan til fuldt ud at gennemføre EU-retten på deres kompetenceområder og at beskytte de rettigheder, som EU-retten tildeler borgerne (direkte anvendelse af EU-retten), hvorved en herimod stridende regel i national ret bliver uvirksom, hvad enten den er yngre eller ældre end EU-normen (EU-rettens forrang for national ret).
Domstolen har ligeledes anerkendt princippet om medlemsstaternes ansvar for tilsidesættelse af unionsretten, som udgør dels et element, der på afgørende måde beskytter de rettigheder, som er tillagt private ved unionsrettens bestemmelser, dels en faktor, der bidrager til en mere omhyggelig gennemførelse af disse bestemmelser i medlemsstaterne. De retskrænkelser, som disse gør sig skyldige i, vil således kunne medføre erstatningspligt, der i visse tilfælde kan få alvorlige konsekvenser for deres offentlige finanser. Desuden vil enhver manglende overholdelse fra en medlemsstats side af unionsretten kunne indbringes for Domstolen, og denne vil, i tilfælde af manglende opfyldelse af en dom, som fastslår et sådant retsbrud, kunne pålægge staten en tvangsbøde og/eller betaling af et fast beløb. Hvis Kommissionen ikke er blevet underrettet om foranstaltningerne til gennemførelse af et direktiv, kan Domstolen imidlertid efter anmodning herom fra Kommissionen pålægge medlemsstaten en økonomisk sanktion allerede fra tidspunktet for afsigelsen af den første traktatbrudsdom.
Domstolen arbejder tæt sammen med de nationale retter, som er unionsrettens ordinære domstole. Enhver national ret, der skal afgøre en tvist med berøring til unionsretten, kan, og skal undertiden, forelægge Domstolen præjudicielle spørgsmål. Domstolen får således lejlighed til at fremlægge sin fortolkning af en unionsretlig regel eller til at kontrollere dens lovlighed.
Udviklingen i Domstolens praksis illustrerer dens bidrag til skabelsen af et retsområde, der angår borgerne, idet den beskytter de rettigheder, som EU-lovgivningen tildeler dem på forskellige områder af deres dagligdag.
\#\#\# Grundsætninger fastslået i retspraksis
Ved en serie domme (begyndende med dommen i sagen Van Gend \& Loos i 1963) indførte Domstolen i sin retspraksis princippet om fællesskabsrettens direkte virkning i medlemsstaterne, som nu gør det muligt for Europas borgere direkte at påberåbe sig bestemmelserne i unionsretten for deres nationale retter.
I forbindelse med indførsel fra Tyskland til Nederlandene skulle transportvirksomheden Van Gend \& Loos betale told, som firmaet fandt stridende mod EØF-traktatens regel om forbud til medlemsstaterne mod at forhøje tolden i deres gensidige handelssamkvem. Sagen rejste spørgsmålet om konflikten mellem en national lovgivning og EØF-traktatens regler. En nederlandsk ret forelagde sagen for Domstolen, og den afgjorde spørgsmålet ved at hævde doktrinen om direkte virkning, således at transportvirksomheden fik en direkte sikkerhed for sine rettigheder i henhold til fællesskabsretten for den nationale ret.
I 1964 blev det i Costa-dommen fastslået, at fællesskabsretten har forrang for intern lovgivning. I denne sag havde en italiensk ret spurgt Domstolen, om en italiensk lov om nationalisering i sektoren for fremstilling og distribution af elektricitet var forenelig med en række regler i EØF-traktaten. Domstolen indførte doktrinen om fællesskabsrettens forrang, som den begrundede med den særlige karakter af Fællesskabets retsorden, der må anvendes ensartet i samtlige medlemsstater.
I 1991 udviklede Domstolen i dommen i sagen Francovich m.fl. et andet grundbegreb, nemlig grundbegrebet om medlemsstatens ansvar over for private for den skade, de måtte lide på grund af, at staten har tilsidesat fællesskabsretten. Siden 1991 står der altså et erstatningssøgsmål til rådighed for de europæiske borgere, som de kan rejse mod den stat, der overtræder en EF-regel.
To italienske borgere, som havde løntilgodehavender hos deres konkursramte arbejdsgivere, havde indledt sag, hvorunder de påberåbte sig den italienske stats passivitet, idet den ikke havde gennemført EF-reglerne om beskyttelse af arbejdstagere i tilfælde af deres arbejdsgivers insolvens. Efter at en italiensk ret havde forelagt sagen for Domstolen, udtalte denne, at det pågældende direktiv tog sigte på at tillægge private rettigheder, som de var blevet afskåret fra at udnytte som følge af, at staten havde udvist passivitet, da den ikke havde gennemført direktivet, og hermed gjorde Domstolen vejen fri for et erstatningssøgsmål mod staten selv.
\#\#\# Domstolens rolle i unionsborgerens liv
Blandt de tusinder af domme, Domstolen har afsagt, har størstedelen, navnlig de, der er afsagt under den præjudicielle procedure, helt åbenbart betydelige virkninger for de europæiske borgeres dagligliv. Heraf skal nævnes nogle stykker som eksempel på de væsentligste af fællesskabsrettens områder.\\
\#\#\#\# Frie varebevægelser
Efter dommen i sagen Cassis de Dijon, der blev afsagt i 1979 vedrørende grundsætningen om frie varebevægelser, kan de erhvervsdrivende til deres land indføre enhver vare med oprindelse i et andet unionsland - på den betingelse, at varen dér er blevet fremstillet lovligt og bragt i omsætning, og at ingen tvingende grund, f.eks. forbundet med beskyttelsen af sundhed og miljø, er til hinder for varens indførsel til forbrugslandet.
\#\#\#\# Fri bevægelighed for personer
Adskillige domme er blevet afsagt på området for fri bevægelighed for personer.
I Kraus-dommen (1993) fastslog Domstolen, at retstillingen for en EF-borger, der er indehaver af et bevis for afsluttet universitetseksamen, som er erhvervet i en anden medlemsstat, og som letter adgangen til en profession eller udøvelsen af en økonomisk virksomhed, er reguleret af fællesskabsretten, endog hvad angår den pågældende borgers forhold til sin hjemstat. Det gælder således, at selv om en medlemsstat kan betinge anvendelsen af eksamensbeviset på sit område af en administrativ godkendelse, må godkendelsesproceduren alene have det formål at klarlægge, om beviset er blevet lovligt udstedt.

Blandt de øvrige domme afsagt på dette område er en af de mest kendte Bosman-dommen (1995), hvori Domstolen efter anmodning fra en belgisk ret tog stilling til spørgsmålet, om fodboldforbunds regler kunne anses for forenelige med arbejdskraftens frie bevægelighed. Domstolen udtalte, at professionel sport er en økonomisk virksomhed, hvis udøvelse ikke kan hindres af reglerne om spillertransfert eller ved en begrænsning af antallet af spillere med statsborgerskab i en andre medlemsstater. Sidstnævnte udtalelse er ved senere domme blevet udvidet til at gælde retsstillingen for professionelle sportsudøvere med oprindelse i tredjelande, der har indgået en associerings- (dommen i sagen Deutscher Handballbund fra 2003) eller partnerskabsaftale (Simutenkov-dommen 2005) med De Europæiske Fællesskaber.
\#\#\#\# Fri udveksling af tjenesteydelser
En dom fra 1989 vedrørende fri udveksling af tjenesteydelser angik en britisk turist, der var blevet overfaldet og alvorligt såret i den parisiske metro. Efter en forelæggelse fra en fransk ret fastslog Domstolen, at den pågældende i sin egenskab af turist havde adgang til tjenesteydelser uden for sit land og principielt var omfattet af det grundlæggende forbud mod forskelsbehandling på grundlag af nationalitet, der er knæsat i fællesskabsretten. Turisten havde følgelig ret til samme erstatning, som en fransk statsborger kunne gøre krav på (Cowan-dommen).
Efter en præjudiciel anmodning fra luxembourgske retter fastslog Domstolen, at en national lovgivning med den virkning, at en forsikringstager får afslag på godtgørelse af udgifter ved tandbehandling med den begrundelse, at denne har fundet sted i en anden medlemsstat, udgør en uberettiget hindring for den frie udveksling af tjenesteydelser (Kohll-dommen, 1998), og at et afslag på godtgørelse af udgifter til køb af briller i udlandet må bedømmes som en uberettiget hindring for de frie varebevægelser (Decker-dommen, 1998).
\#\#\#\# Ligebehandling og sociale rettigheder
En flystewardesse anlagde sag mod sin arbejdsgiver på grund af forskelsbehandling vedrørende den løn, hun oppebar i forhold til sine mandlige kolleger, der udførte samme arbejde. En belgisk ret forelagde sagen for Domstolen, som i 1976 fastslog, at traktatreglen med påbuddet om principiel ligebehandling med hensyn til løn til kvindelige og mandlige arbejdstagere for samme arbejde havde direkte virkning (Defrenne-dommen).

Ved en fortolkning af EF-reglerne vedrørende ligebehandling af mænd og kvinder har Domstolen bidraget til at beskyttelsen af kvinderne mod afskedigelse i forbindelse med barsel. En kvinde, der ikke længere kunne arbejde som følge af besværligheder forbundet med graviditet, blev afskediget. I 1998 kendte Domstolen afskedigelsen stridende mod fællesskabsretten. Afskedigelse af en kvinde under graviditet som følge af fravær, forårsaget af en sygdom på grund af selve graviditeten, er en forbudt forskelsbehandling på grundlag af køn (Brown-dommen).

Med henblik på at sikre arbejdstagernes sikkerhed og sundhed er det nødvendigt, at disse har adgang til betalt årlig ferie. I 1999 rejste den britiske fagforening BECTU tvivl om britiske regler, som fratog arbejdstagere med arbejdskontrakter af kort varighed denne ret med den begrundelse, at reglerne ikke var i overensstemmelse med et EF-direktiv vedrørende fastlæggelse af arbejdstiden. Domstolen nåede til det resultat (BECTU-dommen, 2001), at retten til betalt årlig ferie er en social ret, som direkte er tillagt samtlige arbejdstagere i fællesskabsretten, og at ingen arbejdstager kan afskæres herfra.
\#\#\#\# Grundlæggende rettigheder
Med en udtalelse om, at overholdelsen af grundlæggende rettigheder er en integrerende del af de almindelige retsgrundsætninger, som Domstolen skal søge overholdt, har Domstolen bidraget væsentligt til højere standarder i relation til beskyttelsen af de nævnte rettigheder. Herved tager Domstolen hensyn til de for medlemsstaternes fælles forfatningstraditioner og folkeretlige aftaler om beskyttelse af menneskerettighederne, navnlig den europæiske konvention til beskyttelse af menneskerettigheder, som medlemsstaterne har samarbejdet om eller tiltrådt. Efter ikrafttrædelsen af Lissabontraktaten vil Domstolen kunne anvende og fortolke Den Europæiske Unions charter om grundlæggende rettigheder af 7. december 2000, som i medfør af Lissabontraktaten tillægges samme retsværdi som traktaterne.
Efter talrige terroristattentater mod politiembedsmænd indførte man i Nordirland bevæbning af politistyrkerne. Af hensyn til den offentlige sikkerhed tillod man imidlertid ikke bevæbning (på grundlag af en attest udstedt af vedkommende minister, som ikke kunne anfægtes ved domstolene) af kvinder ansat i politiet. Som følge heraf var der ikke længere mulighed for fuldtidsansættelse af kvinder i det nordirske politi. Efter en præjudiciel forelæggelse fra en ret i Det Forenede Kongerige afgjorde Domstolen, at udelukkelsen af enhver domstolsprøvelse af en attest fra en national myndighed er i strid med princippet om en effektiv domstolsbeskyttelse, der tilkommer enhver, som finder sig ramt af kønsdiskriminering (Johnston-dommen, 1986).
\#\#\#\# Unionsborgerskab
Vedrørende unionsborgerskabet, som ifølge traktaten om Den Europæiske Unions virkemåde tilkommer enhver statsborger i medlemsstaterne, har Domstolen bekræftet, at dette indebærer retten til ophold på en anden medlemsstats område. Således har en mindreårig statsborger i en medlemsstat, som er sygeforsikret og har tilstrækkelige midler til sit underhold, også ret til ophold. Domstolen fremhævede, at fællesskabsretten ikke kræver af den mindreårige, at han selv har de nødvendige midler, og at afslaget på samtidig at meddele den mindreåriges moder, med statsborgerskab i tredjeland, ret til ophold, vil bevirke, at barnets opholdsret bliver uden enhver effektiv virkning (dommen i sagen Zhu og Chen, 2004).\\
I samme dom præciserede Domstolen, at en medlemsstat - selv i det tilfælde, hvor erhvervelse af statsborgerskab i en medlemsstat har til formål at opnå en opholdsret i henhold til fællesskabsretten for en statsborger i en medlemsstat - ikke kan indskrænke virkningerne af tildeling af statsborgerskab i en anden medlemsstat.

\hypertarget{retskilderne}{%
\section{Retskilderne}\label{retskilderne}}

Retskilder, de faktorer, som danner grundlaget for at opnå viden om, hvad der er gældende ret.
Lovgivningen er den primære retskilde, som altid skal tages i betragtning ved fastlæggelse af retsstillingen, men retspraksis er også en vigtig retskilde, især hvis der foreligger en afgørelse fra Højesteret (se præjudikat) eller en afgørelse fra EU-Domstolen eller fra Menneskerettighedsdomstolen vedrørende Den Europæiske Menneskerettighedskonvention.
Andre faktorer, der kan anvendes som retskilder, er fx retssædvaner, aftaler, lovforarbejder, administrativ praksis, administrative retsforskrifter, udtalelser fra Folketingets Ombudsmand samt retsvidenskabens analyser. Herudover anvendes traditionelt også forholdets natur som retskilde.\\
Der kan ikke opstilles en fast prioritering af retskilderne, idet de i konkrete tilfælde alle kan have betydning for, hvad der er gældende ret.
Der er stor forskel på retskildeopfattelsen inden for forskellige retssystemer, se fx case law.

\hypertarget{case-law}{%
\subsection{Case law}\label{case-law}}

Case law, judge-made law, dommerskabt ret. I common law-lande har domstolene en mere vidtgående funktion end den dømmende magt i andre retssystemer. Common law-domstolen skal ikke kun fortolke og anvende lovgiverens retsregler, statutory law, men skaber også sin egen ret, case law eller judge-made law. En domstolsafgørelse har således ikke alene betydning for parterne i en konkret tvist; afgørelsen skaber præcedens, dvs. at en lignende sag i fremtiden med stor sandsynlighed vil blive afgjort på samme måde, den såkaldte stare decisis-doktrin.
Betegnelsen case law er engelsk, af case `tilfælde' og law `lov'.
I England er de afgørelser, der hidrører fra samme eller en højere domstol, bindende. Andre afgørelser har kun vejledende karakter. Der sondres desuden mellem den tidligere afgørelses egentlige begrundelse, ratio decidendi, og andre udtalelser i afgørelsen, obiter dicta. Et obiter dictum er kun vejledende, uanset fra hvilken domstol det måtte hidrøre.
Selv bindende afgørelser bliver fra tid til anden tilsidesat (overruled); fx kan der bag afgørelsen ligge en forældet tankegang, og i 1966 udtalte Englands højeste domstol, House of Lords, at den ikke fremover ville betragte sig som evigt bundet af sine egne tidligere afgørelser.
I USA håndhæves stare-decisis mindre strengt end i England. USA's højesteret har enkelte gange tilsidesat sine tidligere afgørelser, selv i sager om fortolkning af USA's forfatning. Fx gav United States Supreme Court i 1954 sorte elever adgang til skoler og universiteter, som tidligere havde været forbeholdt de hvide, og tilsidesatte herved sin egen ældre afgørelse, der havde anerkendt doktrinen om ``separate but equal'', dvs. at man i undervisningen adskilte sorte og hvide.

\hypertarget{retskildepolycentri}{%
\subsection{Retskildepolycentri}\label{retskildepolycentri}}

Retskildepolycentri er en nyere retsvidenskabelig erkendelse af, at dannelsen af retskilder i forskellige fora i det moderne samfund kan resultere i, at en retskilde kan have forskellig virkning for forskellige retsanvendere. Teorien bryder med den hierarkiske retskildeopfattelse. Teoriens ophavsmand er den danske juraprofessor Henrik Zahle.

\hypertarget{juridisk-metode}{%
\section{Juridisk metode}\label{juridisk-metode}}

Juridisk metode, fremgangsmåde ved stillingtagen til juridiske problemer. Metoden består for det første af en beskrivelse og identifikation af de retskilder, som gyldigt kan inddrages i en juridisk argumentation; for det andet af læren om, hvordan retskilderne fortolkes. Juridisk metode består af 3 hovedelementer:

Et faktum (Hændelsesforløbet fx der er sket økonomisk misbrug af en kortholderens mistede dankort)\\
+\\
Et retsfaktum (Hvilken retsregel i betalingsloven skal anvendes i forhold til tredjemandsmisbruget af Dankortet)\\
=\\
En retsfølge (Afgørelsen, hvem der kommer til at betale for misbruget af dankortet banken eller kortholderen)

Se f.eks. i henhold til bestemmelsen i betalingslovens § 100, stk. 4, nr. 3, hæfter betaleren for op til 8.000 kr. af misbrug, som finder sted som følge af betalerens groft uforsvarlige adfærd.
Begrebet groft uforsvarlig adfærd er ikke nærmere afgrænset i betalingsloven, men traditionelt anvendes begrebet »grov uagtsomhed« som betegnelse for »tilsidesættelse af den agtpågivenhed, som selv skødesløse personer plejer at udvise.« Med anvendelsen af begrebet groft uforsvarlig adfærd er det således præciseret, at grov uagtsomhed i sædvanlig forstand ikke er tilstrækkeligt til at pådrage betaleren hæftelse efter bestemmelsen. Der skal altså mere til. Groft uforsvarlig adfærd må herefter forstås som sløseri, der er præget af ligegyldighed i forbindelse med opbevaring af bl.a. pinkoden. Der skal derfor meget til efter praksis i Pengeinstitutankenævnet, før der statueres groft uforsvarlig adfærd.

\begin{center}\rule{0.5\linewidth}{\linethickness}\end{center}

\emph{PIA 78/2006: »Det forhold, at misbrugeren på klagerens bopæl tilfældigt fik mulighed for at overhøre klageren oplyse sin kode til dankortet til kæresten sammenholdt med, at dankortet opbevaredes i hendes pung, der lå i hendes jakke på bopælen, kan ikke betegnes som groft uforsvarlig adfærd, heller ikke selv om klageren havde givet T adgang til sin bopæl, uanset om hun måtte have kendskab til T's kriminelle baggrund« }

\emph{PIA 205/2005: »Klagerens Visa/Dankort blev opbevaret i en pung, som lå i en jakke, der var anbragt bag disken i klagerens butik og ikke var synlig for kunderne. Selvom det må bebrejdes klageren, at pinkoden til kortet var anført på en seddel, der lå i pungen sammen med kortet, findes klageren efter en samlet vurdering ikke at have udvist en groft uforsvarlig adfærd som omhandlet i lov om visse betalingsmidler § 11, stk. 3, nr. 3. Ankenævnet har herved også lagt vægt på, at det i lovens forarbejder er forudsat, at det udvidede ansvar kun ville kunne gøres gældende i et fåtal af tilfælde«.}

\emph{Selvforskyldt beruselse kan blive betragtet som uforsvarlig adfærd, jf. PIA 281/2013 Spørgsmål om misbrug af kort var muliggjort ved groft uforsvarlig adfærd som følge af beruselse: »Som sagen foreligger oplyst, lægger vi til grund, at klageren ikke ved, hvad han foretog sig fra ca. kl. 00.30 til ca. kl. 02.00, da han vågnede i en bil, som han formoder var en pirattaxa. Vi finder ikke grundlag for at antage, at klagerens tilstand skyldes andre forhold end indtagelse af alkohol. Under disse omstændigheder finder vi, at misbruget af klagerens betalingskort er muliggjort, fordi han var stærkt påvirket af alkohol. Vi finder, at klageren under de beskrevne omstændigheder har udvist groft uforsvarlig adfærd. Vi stemmer derfor for at lade klageren hæfte med 8.000 kr. af det tab, der opstod som følge af den uberettigede brug af kortet, jf. lov om betalingstjenester § 62, stk. 3 nr. 3«.}

\emph{Det er en konkret vurdering, om der bliver statueret groft uforsvarlig adfærd, jf. fx PIA 436/1993: Natten mellem den 12. og 13. juni 1993 blev klageren, medens han opholdt sig på en restauration, frastjålet sit dankort, som var opbevaret i klagerens tegnebog. Klageren anmeldte den 13. juni 1993 kl. 11.10 telefonisk tyveriet til kriminalpolitiet i Sønderborg efter forinden telefonisk at have spærret dankortet ved meddelelse til PBS. Det viste sig efterfølgende, at der ved anvendelse af dankortet og korrekt pinkode den 13. juni 1993 mellem kl. 9.25 og 9.28 var hævet 3 x 2.000 kr. i tre forskellige dankortautomater. Efter de foreliggende oplysninger lagde Pengeinstitutankenævnet til grund, at tyveriet af dankortet var blevet forøvet af to unge piger, som senere blev dømt for tyveri ved den 13. juni 1993 kl. 9.25-9.28 under anvendelse af dankortet at have stjålet de 3 x 2.000 kr. fra dankortautomaterne. Det fremgik af en retsbogsudskrift fra straffesagen, at de sigtede havde forklaret, at den ene af dem, A, snakkede med klageren, mens den anden sigtede, B, tog pungen op af lommen på ham. Herefter gik de ud på toilettet, hvor de tog pengene og kortet. A forklarede videre, at hun havde spurgt klageren om koden, og han havde givet hende den. Han sagde noget om, at han skulle ringe til banken for at få kortet spærret, og hun tilbød at gøre det for ham. Hun lod, som om hun telefonerede til banken, og sagde i den forbindelse til klageren, at hun skulle bruge pinkoden, hvorefter han gav hende den. Klageren indbragte sagen for Ankenævnet med påstand om, at indklagede var tilpligtet at anerkende, at klageren ikke hæftede for de 6.000 kr. Ankenævnet traf følgende afgørelse: »Efter betalingskortlovens § 21, stk. 2, hæfter kortindehaveren uden beløbsbegrænsning for tab, der opstår som følge af andres uberettigede brug af betalingskortet og den dertil hørende personlige hemmelige kode, såfremt kortudstederen godtgør, at kortindehaveren har oplyst koden til den, der har foretaget den uberettigede brug. Ankenævnet finder imidlertid ikke, at bestemmelsen er anvendelig på et tilfælde som det foreliggende. Ankenævnet finder på den anden side, at klageren udviste groft uforsvarlig adfærd ved i det foreliggende tilfælde at oplyse sin PIN-kode. Han hæfter derfor med op til 8.000 kr. for det tab, der opstod som følge af det uberettigede brug af kortet, jf. betalingskortlovens § 21, stk. 3, nr. 2, (nu § 11, stk. 3, nr. 3). Som følge af det anførte bestemmes: Den indgivne klage tages ikke til følge«.}

\begin{center}\rule{0.5\linewidth}{\linethickness}\end{center}

Den juridiske metode indeholder væsentlige elementer af vurdering og skøn og er derfor mindre eksakt end de metoder, der anvendes inden for mange andre fagområder.
I juridisk Ordbog defineres juridisk metode som følger:

\hypertarget{aftaleret}{%
\chapter{Aftaleret}\label{aftaleret}}

\hypertarget{quiz}{%
\section{Quiz}\label{quiz}}

Quiz Aftaleret

\hypertarget{fuldmagter-og-mellemmnd}{%
\chapter{Fuldmagter og mellemmænd}\label{fuldmagter-og-mellemmnd}}

\hypertarget{forbrugeraftaler}{%
\chapter{Forbrugeraftaler}\label{forbrugeraftaler}}

\hypertarget{erstatning-og-den-forsikringsmssige-afdkning}{%
\chapter{Erstatning og den forsikringsmæssige afdækning}\label{erstatning-og-den-forsikringsmssige-afdkning}}

\hypertarget{erstatningsansvar-uden-lovgivning}{%
\section{Erstatningsansvar uden lovgivning}\label{erstatningsansvar-uden-lovgivning}}

Ofte findes der ingen lovgivning, som kan fortælle, hvornår man som privatperson er ansvarlig for en skade. I stedet er der ud af mange års domspraksis udledt en erstatningsregel, der kaldes ``den almindelige erstatningsregel''. Denne regel fastslår, at man er ansvarlig for den økonomiske skade, som man har forvoldt ved uagtsomhed eller med vilje. Reglen hedder også culpa- eller skyldreglen. Culpa, er den grundlæggende betingelse for at pålægge erstatningsansvar uden for kontraktforhold. En skadevolder har udvist culpa, optrådt culpøst, overtrådt culpareglen, hvis han eller hun har handlet enten forsætligt, dvs. med vilje og viden om handlingens elementer, eller uagtsomt. Se f.eks. følgende domme:

\begin{center}\rule{0.5\linewidth}{\linethickness}\end{center}

\emph{FED 2012.2: Landsretten fandt, at en tilskuer, der ved strafbar handling løb ind på fodboldbane og afbrød landskamp, var erstatningsansvarlig for DBU's tab ved, at næste landskamp blev flyttet til en mindre bane samt ved betaling af bøde til UEFA (1.869.269 kr. plus renter). Der forelå ikke egen skyld eller anledning til at reducere ansvaret efter EAL (erstatningsansvarsloven) § 24.}

\emph{FED 2006.78: Ansvarspådragende efter culpareglen, at en person, der kørte i en lånt bil, ikke sikrede sig, at der var olie og vand på bilen, uagtet temperaturlampen lyste og uagtet, at han kort forinden med få dages mellemrum havde været nødsaget til at påfylde olie og vand. (Utrykt)}

\emph{U 2015.572 H (U: Ugeskrift for Retsvæsen): Tobaksselskaber ikke ansvarlige for varigt mén som følge af mangeårigt forbrug af cigaretter.}

\begin{center}\rule{0.5\linewidth}{\linethickness}\end{center}

Man er kun erstatningsansvarlig for skader, der kan gøres op i penge (økonomisk tab, som den skadelidte skal kunne bevise). Ved tingsskader vil det være værdien af eller reparation af det ødelagte. Ved personskader bliver det økonomiske tab beregnet efter reglerne i Lov om erstatningsansvar (erstatningsansvarsloven). Det er f.eks. erstatning for varige mén, tabt arbejdsfortjeneste eller tab af forsørger. Se følgende domme, hvor der i dømmes erstatningsansvar i forskellige situationer:

\begin{center}\rule{0.5\linewidth}{\linethickness}\end{center}

\emph{FED 2012.3: A, der sad uden for golfbanens cafeteria, blev ramt i hovedet af golfbold, da B -- som deltager i en polterabend -- under golfspil kom til at slå bolden for langt. B, der kun én gang tidligere havde prøvet at spille golf og ikke havde modtaget undervisning heri, fandtes erstatningsansvarlig.}

\emph{FED 2009.114: En kvinde, som faldt i søvn som fører af en bil, påkørte og dræbte en modkørende bilist. Hun blev idømt en fængselsstraf på 10 måneder, men hendes kørsel kunne ikke karakteriseres som særlig hensynsløs, hvorfor afdødes ægtefælle ikke havde krav på godtgørelse i medfør af EAL (erstatningsansvarsloven) § 26 a.}

\emph{FED 2008.1: En mandlig stripteasedanser B ansvarlig for øjenskade på deltager A i kvindelig polterabend, da B affyrede konfettirør. Ikke grundlag for at nedsætte erstatningen som følge af egen skyld eller accept af risiko.}

\begin{center}\rule{0.5\linewidth}{\linethickness}\end{center}

\hypertarget{hndeligt-uheld}{%
\section{Hændeligt uheld}\label{hndeligt-uheld}}

Har skadevolderen derimod opført sig som en såkaldt bonus pater familias (latin; `den gode familiefader'), har han eller hun ikke gjort noget forkert og er ikke ansvarlig for skaden. Man skal heller ikke erstatte noget. Det kaldes et hændeligt uheld. Åbner en hotelgæst eksempelvis døren til sit værelse indefra, i det øjeblik en tjener går forbi på gangen med en bakke med glas, og døren rammer bakken, så glassene falder på gulvet og knuses, er der tale om et hændeligt uheld. Uheldet kan ikke bebrejdes hotelgæsten, da han ikke havde mulighed for at undgå det. Han havde opført sig som ``den gode hotelgæst'' og ikke begået nogen fejl. Se afgørelserne fra retspraksis og Forsikringsankenævnet om spørgsmålet om hændelige skader:

\begin{center}\rule{0.5\linewidth}{\linethickness}\end{center}

\emph{U.1961.167 H: At en 11-årig dreng kom til at ramme en anden dreng med en kæp i øjet, blev anset for et hændeligt uheld og derfor ikke noget erstatningsansvar.}

\emph{FED 1998.584: Ejeren af sommerhus til udlejning havde ikke pådraget sig erstatningspligt ved, at en lejers barn kom til skade, da et sofabord, der var en del af sommerhusets møblement, væltede.}

\emph{Ankenævnskendelse i sag nr. 35.936 -- 30.12.94: Klagers ``griben ud efter'' sit 15 måneder gamle barn, der var ved at falde, hvorved cigaretglød ødelagde en sofa, ikke anset som uagtsom adfærd, derfor ikke nogen erstatningspligt for sofaen.}

\emph{Ankenævnskendelse i sag nr. 39.737 -- 22.01.1996: Gæstebudsskade at tabe kaffebakke under forsøg på at undgå at træde på et barn. Ikke erstatningsansvar, men dækket som gæstebudsskade over personens ansvarsforsikring.}

\emph{Ankenævnskendelse i sag nr. 51.809 -- 29. maj 2000: 8-årigt barn faldt over egne ben og væltede vase. Ikke erstatningsansvar, men dækket som gæstebudsskade.}

\emph{FED 2003.1091: Hønseejer var ikke ansvarlig for cyklists tilskadekomst, da cykel på landevej ramte fritgående høne.}

\emph{U 2013.84 V: Motionscyklist, der sammen med andre cyklister kørte i en gruppe, og som efter at være kommet ud i rabatten prøvede at komme op på vejen igen, hvorved han væltede, og der skete sammenstød med bagfrakommende cyklister, havde ikke handlet ansvarspådragende.}

\emph{H.K. af 24. juni 2002. Sag: 206/2002: Museum var ikke ansvarligt for tilskadekomst, da gæst under privat besøg gled i hundeekskrementer og faldt.}

\begin{center}\rule{0.5\linewidth}{\linethickness}\end{center}

\hypertarget{handlet-forkert-eller-undladt-at-handle}{%
\section{Handlet forkert eller undladt at handle}\label{handlet-forkert-eller-undladt-at-handle}}

Har skadevolder handlet forkert -- dvs. er ansvarlig -- er det vigtigt at finde ud af, om skaden er forvoldt ved en simpel uagtsomhed, grov uagtsomhed eller med forsæt (med vilje). Denne vurdering af handlingen eller undladelsen har bl.a. betydning for, om skadevolderens ansvarsforsikring skal betale for skaden hos skadelidte. Se her afgørelserne fra retspraksis:

\begin{center}\rule{0.5\linewidth}{\linethickness}\end{center}

\emph{FED 2016.122: Diskotek erstatningsansvarlig for personskade opstået ved, at diskotekets dørmand gik ud over det nødvendige og forsvarlige ved at skubbe eller kaste skadelidte ud af diskotekets dør.}

\emph{Ø.L.D. af 12. maj 2005. Sag: 20. afd., a.s. nr. B-2581-04: Værtinde ifaldt erstatningsansvar, fordi hun bar et sofabord ned ad en trappe iført højhælede sko og uden at fjerne noget nips, der indskrænkede trappearealet, hvilket forårsagede, at hun tabte bordet, som ramte en gæst i hovedet. (Utrykt)}

\begin{center}\rule{0.5\linewidth}{\linethickness}\end{center}

\hypertarget{den-gode-familiefader}{%
\subsection{Den gode familiefader}\label{den-gode-familiefader}}

Når en domstol bedømmer, om skadevolder har handlet forkert, sammenligner domstolen skadevolderens handling med, hvordan en ``bonus pater familias'' ville have handlet i samme situation.
Bonus pater familias (latin for den gode familiefader) er en fiktiv person, som aldrig begår fejl, fordi han eller hun altid tænker sig grundigt om, før denne handler. I nogle situationer vil denne fiktive person være ``den gode forældre'', ``det gode barn'', ``den gode lærer'', ``den gode håndværker'' osv. Begrebet repræsenterer en uagtsomhed en adfærd, der afviger fra de adfærdsnormer, der gælder inden for det pågældende område.
Kommer en domstol til det resultat, at en skadevolder har handlet anderledes, end en bonus pater familias ville have gjort i samme situation, har skadevolderen handlet forkert og vil blive pålagt et erstatningsansvar.
Det skal bevises af skadelidte (kaldet en ''ligefrem bevisbyrde''). Det er hovedreglen i dansk ret; Den der vil gøre noget gældende, har også bevisbyrden herfor.
For at kunne få erstatning fra en skadevolder skal skadelidte ifølge retspraksis (domstolene) bevise følgende over for retten:

\begin{itemize}
\tightlist
\item
  Der er sket en skade og lidt et tab.\\
\item
  Det er skadevolders skyld -- skadevolder har handlet uagtsomt eller med forsæt.\\
\item
  Der er årsagssammenhæng (kausalitet) mellem den skete skade og det lidte tab, dvs. at tabet er en direkte følge af skaden.\\
\item
  Der er påregnelighed (adækvans) mellem skaden og tabet, dvs., at skadevolder burde kunne forudse, at skaden ville ske.
\end{itemize}

Kan skadelidte ikke bevise det, er det ikke muligt at få erstatning fra en skadevolder eller dennes ansvarsforsikring:

\begin{center}\rule{0.5\linewidth}{\linethickness}\end{center}

\emph{FED 2007.111: Forsikringstager havde ikke dokumenteret, at der var årsagsforbindelse mellem nogle anførte lidelser og et færdselsuheld.}

\begin{center}\rule{0.5\linewidth}{\linethickness}\end{center}

\hypertarget{ansvar-for-ikke-at-gre-noget}{%
\subsection{Ansvar for ikke at gøre noget}\label{ansvar-for-ikke-at-gre-noget}}

Det er ikke kun handlinger, man kan blive erstatningsansvarlig for. I visse tilfælde kan man også blive erstatningsansvarlig for sine undladelser. Det sker oftest i situationer, hvor der er pligt til at handle. Eksempler er husejeren, der ikke gruser et isglat fortov, eller forældre, der ikke holder øje med deres børn.

\hypertarget{ansvar-for-psykiske-personskader}{%
\subsection{Ansvar for psykiske personskader}\label{ansvar-for-psykiske-personskader}}

Udover at der kan tilkendes erstatningsansvar ved tingsskader samt fysiske personskader, så kan der være erstatningspligt for psykiske personskader. Se her nævnte domme:

\begin{center}\rule{0.5\linewidth}{\linethickness}\end{center}

\emph{U.2012.524 H: A blev sygemeldt efter et begivenhedsforløb på en personaleweekend og rejste krav om godtgørelse for svie og smerte mod kommunen K som arbejdsgiver. Begivenhedsforløbet vedrørte en følelsesladet drøftelse i plenum af A's sygdomsfravær, som ifølge A endte med, at hun i overværelse af sine kollegaer reelt blev fyret. K bestred ikke, at A var blevet påført en psykisk skade som følge af begivenhedsforløbet, og parterne var enige om opgørelsen af godtgørelseskravet. Højesteret udtalte bl.a., at udtrykket ``personskade'' i erstatningsansvarslovens § 1 må forstås i overensstemmelse med dansk rets almindelige erstatningsregler, og den psykiske skade, som A var blevet påført, var omfattet af bestemmelsen. Højesteret fandt endvidere, at personaleweekenden i forhold til A blev afviklet på en uforsvarlig måde, og at dette var ansvarspådragende for K som arbejdsgiver. Den uforsvarlige afvikling af personaleweekenden medførte en betydelig forøgelse af risikoen for en psykisk skade hos en medarbejder, som ledelsen må have forstået befandt sig i en psykisk anspændt situation, og Højesteret fandt, at A's psykiske skade var en påregnelig følge. A var direkte skadelidt som følge af den ansvarspådragende adfærd, og der var ikke grundlag for at anse den psykiske skade for at falde uden for, hvad der var omfattet af kommunens erstatningspligt. K skulle herefter betale 50.000 kr. til A for svie og smerte. Landsretten havde frifundet K.}

\emph{U 2010.1609 H: Psykisk personskade anset for omfattet af ulykkesbegrebet i ulykkesforsikring}

\begin{center}\rule{0.5\linewidth}{\linethickness}\end{center}

\hypertarget{simpel-uagtsomhed}{%
\subsection{Simpel uagtsomhed}\label{simpel-uagtsomhed}}

Ved simpelt uagtsomhed (culpa levis), har skadevolder handlet mere skødesløst end en bonus pater familias -- den lille dagligdags uagtsomhed -- er skadevolder erstatningsansvarlig. Simpel uagtsomhed er en uagtsomhed som ikke kan betegnes som grov. Culpareglen omfatter som hovedregel begge former for uagtsomhed. Forsikringsankenævnet og domstolene foretager et konkret skøn, om skadevolder har handlet simpelt uagtsomt. Hvis eksempelvis en fodgænger træder ud på vejbanen uden at se sig for og rammer en forbikørende cyklist, der vælter, kan uheldet bebrejdes fodgængeren, der har været mere skødesløs end ``den gode fodgænger''. Derfor er fodgængeren erstatningsansvarlig for cyklistens økonomiske tab. Se følgende sager fra retspraksis og Forsikringsankenævnet, om spørgsmålet, om der er simpel uagtsomhed hos skadevolder:

\begin{center}\rule{0.5\linewidth}{\linethickness}\end{center}

\emph{U 1915.242 H: (Uagtsom Brandstiftelse). En Tiltalt færdedes en Nat i et straatækket Udhus, hvor der henlaa Fourage og Halm, med en to Tommer lang; Lysestump, som han under sit Ophold i Udhuset tændte. Under sit Ophold i Udhuset røg han derhos Cigaret. Umiddelbart efter at Tiltalte havde forladt Udhuset, viste det sig, at der var Ild i dette, der nedbrændte. Antaget, at Tiltalte havde foraarsaget Ilden, og at der forelaa Tilsidesættelse af almindelig Forsigtighed.}

\emph{Ankenævnskendelse i sag nr. 03.256 -- 03.09.79: Da forsikringstageren, som under pasning af skadelidtes blomster havde forvoldt skade på dennes gulvtæppe, ikke fandtes at have udvist et så uforsvarligt forhold, at han ville kunne gøres ansvarlig for skaden, var selskabet berettiget til at afslå at erstatte denne. (Utrykt)}

\emph{Ankenævnskendelse i sag nr. 03.308 -- 03.09.79: Da forsikringstageren, som under arbejde i sin svigerfars lejlighed forvoldte skade på el-hovedkablet, ikke fandtes at have udvist et så uforsvarligt forhold, at han ville kunne gøres ansvarlig for skaden, var selskabet berettiget til at afslå at erstatte denne (utrykt).}

\begin{center}\rule{0.5\linewidth}{\linethickness}\end{center}

\hypertarget{grov-uagtsomhed}{%
\subsection{Grov uagtsomhed}\label{grov-uagtsomhed}}

Grov uagtsomhed (latin; culpa lata) er en betydelig form for uagtsomhed. Nyere retspraksis lægger vægt på, om skadevolderens adfærd indebar en ''indlysende fare'', for den indtrådte skade. Skadevolderens bevidsthed om faren kan indgå i vurderingen af, om uagtsomheden er grov. Går en fodgænger over for rødt og vælter en cyklist, er fodgængeren nu mere uforsigtig end selv en skødesløs fodgænger. Det kan betegnes som grov uagtsomhed, og fodgængeren er erstatningsansvarlig. Det er en skønsmæssig vurdering, om domstolene og Ankenævnet for Forsikring anser en person for at have handlet groft uagtsomt, se her nævnte domme og ankenævnsafgørelser:

\begin{center}\rule{0.5\linewidth}{\linethickness}\end{center}

\emph{U 1998.1693 H: Opbevaring af nøglen til sikringsboksen i et auktionshus på et kontor i samme bygning var grov uagtsomhed.}

\emph{U 1999.1706 H: Død ved fald på ca. 6 meter fra et tag ikke omfattet af ulykkesforsikring, da faldet var fremkaldt ved forsikredes forsæt eller grove uagtsomhed.}

\emph{U.1993.955 V: Da kortvarig efterladelse af kuffert ved bagagebånd i lufthavn ikke var groft uagtsom, var tyveri af kufferten dækket af tyveriforsikringen.}

\emph{FED 2013.9: Den 16-årige A ville ved ungdomsfest hjælpe værtinden med at tænde op i en udendørs pejs. I den forbindelse hældte han væske fra en dunk, som også værtinden havde benyttet i sit forsøg på at tænde op i pejsen, direkte ind i pejsen. Herved slog ilden tilbage og antændte A's tøj samt dunken, som han i panik kastede fra sig. Den brændende dunk ramte en trækonstruktion og huset nedbrændte. Det viste sig, at væsken var benzin. Landsretten fandt, at A havde handlet groft uagtsomt, hvorfor hans ansvar ikke bortfaldt i medfør af erstatningsanvarslovens (EAL) § 19.}

\emph{FED 2010.93: Forsikringsselskab fik ikke medhold i, at en 13-årig dreng udviste grov uagtsomhed da han startede en bygningsbrand ved at tænde en lighter samtidig med, at en kammerat to meter derfra hældte benzin på en knallert. Tillagt betydning bl.a., at benzindampe har en forholdsvis svag lugt og løber langs gulvet, og at der er store individuelle forskelle på menneskers lugtesans.}

\emph{FED 2001.2121: Passager, som blev dræbt under bilkørsel med beruset fører, fandtes at have handlet groft uagtsomt, hvorfor hans livsarvinger ikke var berettiget til dødsfaldsdækning fra ulykkesforsikring.}

\emph{FED 2001.2255: Bilist, som kørte ud foran tog i jernbaneoverskæring uden at være opmærksom på lys- og lydsignaler, havde handlet groft uagtsomt. DSB var derfor berettiget til at reducere personskadeerstatningen med 1/3.}

\emph{FED 1998.224: 16-årig rulleskøjteløber havde udvist grov uagtsomhed ved at køre ud på kørebanen foran bil. Under henvisning til alder, til at handlingen var udført i kådhed samt til områdets karakter, fandtes der dog ikke grundlag for at nedsætte personskadeerstatningen.}

\emph{FED 2018.01 Ø.L.D. af 11. januar 2018. Sag: 20. afd. nr. B-583-17: Den 14-årige A satte ild til papir i en affaldscontainer i en skolegård og forlod herefter stedet med den 13-årige B, der forholdt sig passivt til A's aktiviteter. Skolen brændte, og forsikringsselskabet F afholdt skadeudgifter på over 10 mio. kr. F gjorde regres med krav på 3 mio. kr. over for A og B. A fandtes at have handlet groft uagtsomt, men erstatningen blev reduceret til 1 mio. kr. efter EAL § 24a. B havde ikke handlet groft uagtsomt og blev derfor frifundet. {[}Processbevilingsnævnet har givet tilladelse til anke til Højesteret{]}.}

\emph{Ankenævnskendelse i sag nr. 20.217 -- 26.08.87J: Kørsel på motorcykel med 1 hånd ikke grov uagtsomhed i familie-/indboforsikring, og selskabet skulle dække beskadiget tøj efter færdselsskade. (utrykt).}

\emph{Ankenævnskendelse i sag nr. 48.636 -- 12. april 1999: Hasarderet kørsel under flugt fra politiet anset for grov uagtsomhed.}

\emph{Ankenævnskendelse i sag nr. 63.558 -- 14. februar 2005: Groft uagtsomt at efterlade ulåst bil med nøgler i tændingslås i flere timer på privat vej, men forholdsvis tæt på befærdet vej. (Utrykt)}

\begin{center}\rule{0.5\linewidth}{\linethickness}\end{center}

\hypertarget{forst-med-vilje}{%
\subsection{Forsæt (med vilje)}\label{forst-med-vilje}}

Forsæt, er en handling der er foretaget med vilje, er ikke i dansk ret en betingelse for at ifalde erstatningsansvar. At skadevolderen har udvist uagtsomhed, dvs. tilsidesat den agtpågivenhed, som kræves på det pågældende område, er som hovedregel tilstrækkeligt til at pålægge erstatningsansvar, Kaster en person en sten efter en cyklist for at ramme denne, og det lykkes, er skaden på cyklisten lavet med vilje -- med forsæt. Skadevolderen er selvfølgelig også her erstatningsansvarlig. Hovedreglen er i dansk ret, at kun forsætlige forhold er strafbare, dog en vigtig undtagelse uagtsom drab. Ved forsætlige handlinger afviser skadevolders ansvarsforsikring at udbetale erstatning, se her nævnte sager om vurderingen om skadevolder har handlet med forsæt -- med vilje:
Speak, billeder eller video:

\begin{center}\rule{0.5\linewidth}{\linethickness}\end{center}

\emph{Forsikringsankenævnskendelse i sag nr. 15.569 -- 26.04.85: Skade på skolelokaler, forårsaget af 15-årig dreng ved udtømning af to pulverslukkere, anset forsætligt forvoldt, og derfor ikke dækket af ansvarsforsikringen. (utrykt)}

\begin{center}\rule{0.5\linewidth}{\linethickness}\end{center}

Ved simpel og grov uagtsomhed er det hovedreglen, at en ansvarsforsikring dækker (se dog afsnittet nedenfor om generelt bortfald af ansvar). Forsætlige skader dækker forsikringen ikke, med mindre skadevolder er under 14 år:
Speak, billeder eller video:

\begin{center}\rule{0.5\linewidth}{\linethickness}\end{center}

\emph{H.D. 19 december 2016 i sag 235/2015 (1. afdeling): Skadelidte, der var sindssyg, forsøgte at begå selvmord ved at køre ind i en modkørende lastbil. Retten til erstatning for personskade bortfaldt som følge af forsætlig medvirken.}

\begin{center}\rule{0.5\linewidth}{\linethickness}\end{center}

\hypertarget{objektivt-ansvar}{%
\subsection{Objektivt ansvar}\label{objektivt-ansvar}}

I dansk ret kan der også gælde et såkaldt ``objektivt ansvar'', hvorefter skadevolderen pålægges erstatningsansvar, selv om der ikke er handlet uagtsomt (uden skyld). Med andre ord er der situationer, hvor man kan blive ansvarlig, selv om skaden sker ved et hændeligt uheld. Objektivt ansvar er bl.a. lovfæstet i produktansvarsloven, jernbaneloven, luftfartsloven, lov om drift af nukleare anlæg, søloven, hundeloven, færdselsloven og lov om formidling af fast ejendom:

\begin{center}\rule{0.5\linewidth}{\linethickness}\end{center}

\emph{FED 1997.92: I medfør af (dagældende) lov om omsætning af fast ejendom § 24 (nu § 47) måtte en ejendomsmægler på objektivt grundlag godtgøre en forbruger (sælger) forskellen mellem det beregnede provenu og et korrekt beregnet provenu, selv om forskellen skyldtes en forkert oplysning om kursen på et lån fra et realkreditinstitut, som havde oplysningen fra Københavns Fondsbørs officielle kursliste.}

\begin{center}\rule{0.5\linewidth}{\linethickness}\end{center}

Men det objektive ansvar kan også være ulovhjemlet (ikke fastsat i loven). Domstolene har i visse tilfælde pålagt skadevoldere objektivt ansvar uden lovhjemmel, når skaden er indtrådt som følge af materialesvigt, f.eks. skader, der er forvoldt af brud på fjernvarmerør, el og gasledninger mv.

\begin{center}\rule{0.5\linewidth}{\linethickness}\end{center}

\emph{FED 2014.77: Bygherre og entreprenører var erstatningsansvarlig for udgravnings- og funderingsarbejde, der gjorde mur på nabogrund ustabil med efterfølgende sammenstyrtning til følge. Ved sammenstyrtningen skete der skader på tilgrænsende ejendom, hvorved denne bygnings ejer blandt andet led et huslejetab. Murens ejer havde inden sammenstyrtningen accepteret, at muren kunne fjernes, hvorfor murens værdi ikke skulle erstattes.}

\emph{FED 2004.507: Kommune og underentreprenør solidarisk ansvarlige for skade på nedgravede kabler, som under udboring til stikledninger blev ødelagt. Ligedeling af ansvaret i det indbyrdes forhold.}

\begin{center}\rule{0.5\linewidth}{\linethickness}\end{center}

\hypertarget{generelt-bortfald-af-ansvar}{%
\section{Generelt bortfald af ansvar}\label{generelt-bortfald-af-ansvar}}

Selv om skadevolder har handlet forkert -- været uagtsom -- er der nogle situationer, hvor skadevolder alligevel ikke er erstatningsansvarlig.
Er den skade, der er sket, i forvejen dækket af en tingsforsikring eller en driftstabsforsikring, er skadevolder ikke erstatningsansvarlig. Det gælder dog kun, hvis skaden ikke er forvoldt med vilje eller ved grov uagtsomhed. Det fremgår af Lov om erstatningsansvar (Lovbekendtgørelse nr. 1070 af 24. august 2018). Smadrer en dreng eksempelvis naboens glasrude, der er forsikret, ved en simpel uagtsomhed, er drengen ikke erstatningsansvarlig. Vinduet bliver betalt af glasforsikringen.
Lov om erstatningsansvar fastslår også, at hvis staten, en kommune eller en anden offentlig institution er selvforsikrende, skal de erstatningsmæssigt betragtes på samme måde, som hvis de havde tegnet forsikring. Det betyder, at der ikke er forskel på, om drengen ved en simpel uagtsomhed, smadrer naboens glasforsikrede rude eller en rude i kommunens skole, hvor kommunen er selvforsikret. I ingen af tilfældene er drengen erstatningsansvarlig. For kommunens vedkommende betyder det, at den selv må betale ruden.
Man kan ikke blive erstatningsansvarlig over for sig selv. De ting, som man ødelægger af sine egne ting, udløser ikke et erstatningsansvar, som der kan dækkes på ens ansvarsforsikring. Se f.eks. her nævnte sag:

\begin{center}\rule{0.5\linewidth}{\linethickness}\end{center}

\emph{FED 2003.836: Som følge af identitet mellem forsikringstageren og ejeren af et sommerhus, var der ikke dækning på en erhvervsansvarsforsikring for skade forvoldt af forsikringstagerens ansatte under udførelse af entreprise på sommerhuset.}

\begin{center}\rule{0.5\linewidth}{\linethickness}\end{center}

\hypertarget{erstatningskravet-og-erstatningsbelbet}{%
\section{Erstatningskravet og erstatningsbeløbet}\label{erstatningskravet-og-erstatningsbelbet}}

I Lov om erstatningsansvar kan man læse, hvad der kan kræves i erstatning af en erstatningsansvarlig skadevolders forsikringsselskab. Man kan som skadelidt kræve erstatning for følgende:

\begin{itemize}
\tightlist
\item
  Udgifter til behandling
\item
  Tabt arbejdsfortjeneste
\item
  Svie og smerte
\item
  Varigt mén
\item
  Tab af erhvervsevne
\item
  Dødsfald
\item
  Skader på ejendele
\item
  Advokatomkostninger.
\end{itemize}

\hypertarget{anmeldelse-af-erstatningskrav}{%
\subsection{Anmeldelse af erstatningskrav}\label{anmeldelse-af-erstatningskrav}}

Skadelidte skal henvende sig til den, der er skyld i skaden. Har skadevolder en ansvarsforsikring, kan vedkommende anmelde skaden til sit eget forsikringsselskab. Har skadevolderen ingen forsikring, må skadevolderen selv betale. Skadelidte må eventuelt rejse sit krav om erstatning gennem en retssag.

\hypertarget{sadan-bliver-man-erstatningsansvarlig-i-relation-til-ansvarsforsikringen}{%
\subsubsection{Sådan bliver man erstatningsansvarlig i relation til ansvarsforsikringen}\label{sadan-bliver-man-erstatningsansvarlig-i-relation-til-ansvarsforsikringen}}

Flere forsikringer indeholder automatisk en ansvarsforsikring, der dækker, hvis en af de forsikrede bliver erstatningsansvarlig for en skade. For mange er det svært at vurdere, hvornår de er ansvarlige for en skade, og hvornår ansvarsforsikringen dækker. Derfor vil reglerne for, hvornår der er ansvar, kort blive gennemgået nedenfor.

Har man forvoldt en skade, er det naturligt, at man føler skyld og mener, at skadelidte skal have erstatning, enten fra en selv eller fra det forsikringsselskab, hvor man er ansvarsforsikret.

At føle skyld for en skade er ikke det samme som, at man juridisk er ansvarlig, og at skadelidte dermed har ret til erstatning. I nogle tilfælde vil det fremgå af lovgivningen, om man er erstatningsansvarlig, mens det i tilfælde, hvor der ikke er lovgivet, vil være retspraksis (domstolene), der afgør, om man er erstatningsansvarlig.

\hypertarget{udgifter-til-behandling}{%
\subsubsection{Udgifter til behandling}\label{udgifter-til-behandling}}

Man kan få erstattet de udgifter til behandling, som ikke bliver betalt fra anden side (f.eks. den offentlige sygesikring eller sin egen ulykkesforsikring). Det kan være udgifter til medicin, fysioterapi, kiropraktorbehandling, hjælpemidler og hjemmehjælp, som lægen har ordineret, eller rimelige udgifter til transport til og fra et behandlingssted.
Fremtidige helbredelsesudgifter bliver erstattet med et kapitalbeløb, der udbetales på én gang og udregnes efter regler, der er fastsat i loven.

\begin{itemize}
\tightlist
\item
  Erstatning for tabt arbejdsfortjeneste
\item
  Erstatning for erhvervsevnetab, hvis denne udbetales som en løbende ydelse
\item
  Renter, uanset om beløbet, de vedrører, er skattepligtigt eller ej. Erstatningen er skattefri, men der skal betales skat af eventuelle renter.
\end{itemize}

\hypertarget{tabt-arbejdsfortjeneste}{%
\subsubsection{Tabt arbejdsfortjeneste}\label{tabt-arbejdsfortjeneste}}

Har man tab af indtægt som følge af uheldet, kan man kræve hele tabet erstattet. Man får erstatning fra det tidspunkt, hvor uheldet skete, og indtil man kan begynde at arbejde igen. Medfører skaden, at man vil få et varigt erhvervsevnetab, får man erstatning frem til det tidspunkt, hvor det er muligt midlertidigt eller endeligt at skønne over ens fremtidige erhvervsevne.

\begin{center}\rule{0.5\linewidth}{\linethickness}\end{center}

\emph{U 2009.485 H: Erstatning for fremtidig tabt arbejdsfortjeneste efter ulykke, der forsinkede studerendes uddannelsesforløb.}

\begin{center}\rule{0.5\linewidth}{\linethickness}\end{center}

\hypertarget{svie-og-smerte}{%
\subsubsection{Svie og smerte}\label{svie-og-smerte}}

Godtgørelse for svie og smerte ydes som kompensation for det ubehag og de smerter, som et uheld medfører. Godtgørelsen kan kræves for perioden fra tidspunktet for uheldet, og normalt så længe man er sygemeldt. Godtgørelsen bliver givet efter en fast takst.

\hypertarget{godtgrelse-for-varigt-men}{%
\subsubsection{Godtgørelse for varigt mén}\label{godtgrelse-for-varigt-men}}

Hvis man får varige gener i sin dagligdag på grund af uheldet, kan man kræve godtgørelse for varigt mén.

Størrelsen af ens mén angives i ménprocent. Det er et lægeligt skøn, der afgør, hvor stort ens mén bliver. Det skal være mindst 5 procent, for at man kan få godtgørelse. Er man fyldt 40 år, bliver godtgørelsen reduceret i forhold til ens alder. Godtgørelsen for varigt mén har intet med ens (evt. tidligere) erhverv at gøre. Derfor vil man måske opleve, at man sagtens kan passe et kontorjob, selv om man har fået erstatning for varigt mén, hvorimod man måske ville have svært ved at passe et job, der var fysisk hårdere.

\hypertarget{erhvervsevnetabserstatning}{%
\subsubsection{Erhvervsevnetabserstatning}\label{erhvervsevnetabserstatning}}

Hvis uheldet betyder, at man mister mindst 15 procent af ens evne til at erhverve indtægt ved arbejde, kan man få erstatning for tab af erhvervsevne. Erstatningens størrelse afhænger af ens løn lige før uheldet, ens alder på uheldstidspunktet og størrelsen af ens erhvervsevnetab. Erstatningen bliver udregnet efter regler fastsat i loven.

Var man fyldt 30 år på tidspunktet for uheldet, bliver ens erstatning nedsat i forhold til ens alder.

Man kan også kræve erstatning, selv om man ikke har nogen egentlig indtægt. For børn, der normalt ingen indtægt har, bliver erstatningen fastsat ved, at ménprocenten bliver ganget med en ``normalårsløn'', der er fastsat i loven. Dette resultat bliver herefter ganget med 10. For hjemmearbejdende og studerende bliver erstatningen fastsat ud fra et skøn over den økonomiske værdi af deres arbejdskraft på det tidspunkt, hvor de kom til skade.

Erhvervsevnetabserstatningen bliver udbetalt på én gang.

\hypertarget{ddsfald}{%
\subsubsection{Dødsfald}\label{ddsfald}}

Der gives erstatning for rimelige begravelsesudgifter. En efterladt ægtefælle eller samlever har desuden krav på et såkaldt ``overgangsbeløb''. Dette beløb skal gøre den økonomiske overgang fra gift/samlevende til enlig nemmere. Beløbets størrelse er fastsat i loven. Der kan dog ikke både gives erstatning for begravelsesudgifter og betales overgangsbeløb.

\hypertarget{afdde-havde-forsrgerpligt}{%
\subsubsection{Afdøde havde forsørgerpligt}\label{afdde-havde-forsrgerpligt}}

Forsørgertabserstatning til ægtefælle eller samlever udgør 30 procent af den erstatning, som afdøde ville have fået udbetalt ved fuldstændigt tab af erhvervsevnen. Der er dog fastsat et minimum- og et maksimumbeløb.

Efterlevende børn får en erstatning, der svarer til summen af de børnebidrag, som afdøde på skadetidspunktet kunne være pålagt at betale. Erstatningen bliver fordoblet, hvis afdøde var eneforsørger.

\hypertarget{ansvaret-og-dkningen-pa-ansvarsforsikringen}{%
\section{Ansvaret og dækningen på ansvarsforsikringen}\label{ansvaret-og-dkningen-pa-ansvarsforsikringen}}

Følgende betragtninger og oplysninger om det private ansvar, husejeransvaret, ansvaret for ens hund, skader på arbejdet samt skader ved udøvelse af sportsaktiviteter bygger hovedsageligt på Forsikringsankenævnets oplysninger i ankenævnets ''klageguide'' samt Forsikringsankenævnets praksis, herunder på retspraksis, samt viden fra Forsikringsoplysningen.

\hypertarget{privatansvarsforsikring}{%
\subsection{Privatansvarsforsikring}\label{privatansvarsforsikring}}

Privatansvarsforsikringen er normalt er en del af ens familie/indboforsikring. Forsikringen dækker den erstatning, man skal betale, hvis man som privatperson forvolder skade på andre personer eller på andres genstande. Der er også knyttet en ansvarsforsikring til en række andre forsikringer, bl.a. en bygningsforsikring.

Privatansvarsforsikringen dækker kun, hvis man juridisk set er erstatningsansvarlig for den indtrufne skade. Forsikringen dækker således normalt ikke, hvis man ønsker at betale erstatning ud fra et ``rimelighedssynspunkt'' eller ud fra ens moralske opfattelse.

Efter dansk ret er man juridisk set ansvarlig for en skade, hvis man har udvist en uforsvarlig adfærd (handling eller undladelse), hvis adfærden kan tilregnes en som forsætlig eller uagtsom, og hvis den skade, som man har forvoldt, må anses for at være en forventelig (påregnelig) følge af ens adfærd. Denne såkaldte culparegel står ikke skrevet i nogen lov, men er udviklet gennem en langvarig retspraksis.

Hændelige skader er man ikke ansvarlig for, se f.eks. her fra retspraksis:

\begin{center}\rule{0.5\linewidth}{\linethickness}\end{center}

\emph{FED 2018.05 V.L.D. af 6. juli 2018. Sag: B-1381-17: 2-årigt barn A, der blev passet af gudmor B, kom alvorligt til skade, da A fik fingre i skål med skoldhed suppe, der væltede ned over A. A og B opholdt sig i et køkken, og B havde vendt ryggen til A i ca. 2 sek. Det var ikke oplyst, hvor på køkkenbordet skålen var placeret, og A havde herefter ikke bevist, at B havde udvist ansvarspådragende uagtsomhed. }

\emph{I sag 86059 fandt Forsikringsnævnet eksempelvis, at klageren havde handlet ansvarspådragende ved at lade en veninde til sin datter benytte en trampolin, hvis sikkerhedsnet ikke var monteret korrekt. Nedenfor ses desuden Forsikringsankenævnets sag 88426 om en 12-årig dreng, der skubber en voksen i ryggen, samt Forsikringsankenævnets sager 87739, 86992, 86059 og 85472:}

\emph{Sag 88426: Klager over afslag på dækning for en ansvarsskade, hvor klagers 12-årige søn var skadevolder. Efter afsluttet fodboldspil, hvor de var på vej ud for at samle bolde ind, gav sønnen faderen et hårdt skub i ryggen, så han faldt forover og beskadigede knæet. Selskabet henviste til, at sønnen ikke var ansvarlig for skaden. Nævnet fandt, at sønnen havde udvist en adfærd, som kunne tilregnes ham som uagtsom. Klagers tilskadekomst var derfor omfattet af hjemforsikringens dækning for ansvar ved personskade. Nævnet lagde vægt på, at skubbet ikke kunne anses for at være sket under et igangværende fodboldspil eller under farlig leg. Klager medhold.}

\emph{Sag 87739: Klager over afslag på at dække en ansvarsskade forvoldt af klagers søn. Sønnen beskadigede gulvet i sit værelse på efterskolen, da han tabte et tændt heksehyl, som han ville smide ud af vinduet. Forsikringen dækkede ikke ansvar på skader på ting, som den sikrede havde i sin varetægt. Nævnet fandt derfor ikke at kunne kritisere selskabets afgørelse. Selskab medhold.}

\emph{Sag 86992: Klager over afslag på dækning for en anmeldt ansvarsskade. Klager anmeldte, at der var sket skade på hendes nabos gulv i forbindelse med, at hun havde passet naboens hus og have og ikke havde fået lukket tilstrækkeligt for vandet efter vanding af blomster. Herved var vandet løbet gennem en haveslange dryppet ned i en spand, hvor haveslangens frie ende var placeret, hvorefter vandet løb ud på gulvet. Selskabet anførte, at der var tale om en hændelig skade. Nævnet fandt, at skaden opstod ved klagers udførelse af en vennetjeneste for naboen, hvor hun ikke fik lukket forsvarligt for vandet. Nævnet fandt, at klager herved havde handlet ansvarspådragende, og lagde bl.a. vægt på, haveslangen var placeret i stuen, og at huset henstod ubeboet på tidspunktet for skadens indtræden. Det forhold, at klagers kortfattede mail af 12/8 2014 kunne give anledning til usikkerhed om, hvem der placerede spanden i stuen, kunne efter nævnets opfattelse ikke føre til andet resultat. Selskabet skulle derfor anerkende, at klager var erstatningsansvarlig. Klager medhold.}

\emph{Sag 86059: Klager over afvisning af dækning for en knæskade, som klagers datters veninde (12 år) pådrog sig under et hop på klagers trampolin den 3/6 2013. Selskabet afviste dækning med henvisning til, at der var tale om et hændeligt uheld, og at klager ikke var ansvarlig for den skete skade. Nævnet fandt, at klager havde handlet ansvarspådragende ved at lade datterens veninde benytte en trampolin med et sikkerhedsnet, der ikke var monteret korrekt. Nævnet fandt det godtgjort, at den mangelfulde montering af sikkerhedsnettet var medvirkende årsag til, at venindens fod satte sig fast, så hun kom til skade. Selskabet skulle derfor anerkende, at klager var ansvarlig for de skader, som veninden måtte have pådraget sig. Det forhold, at skadelidte og skadelidtes mor var bekendt med sikkerhedsnettets tilstand, kunne ikke føre til andet resultat. Klager medhold.}

\emph{Sag 85472: Klager over afvisning af dækning for en ansvarsskade over en familieforsikring. Klager havde etableret et stort hul i gulvet i sin ejerlejlighed med henblik på at opsætte vindeltrappe. En maler faldt ned gennem hullet og kom til skade. Selskabet afviste ansvarsdækning med henvisning til, at der var tale om et grundejeransvar, som ikke var dækket af forsikringen. Nævnets flertal fandt, klagers eventuelle ansvar var et culpaansvar uden for kontrakt, og at han havde ageret som privatperson. Det eventuelle ansvar var derfor dækningsberettigende. Betingelsernes punkt 8.1.4 vedrørende grundejeransvar kunne alene anses for et tillæg til, og ikke en indskrænkning af, den dækning, der ydedes efter hovedreglen i punkt 8.1.1 om ansvar som privatperson. Mindretallet fandt, at betingelsernes 8.1.4 udtømmende gjorde op med dækningen for grundejeransvar, og at den anmeldte skade ikke kunne henføres herunder. Mindretallet bemærkede, at klager kunne have tegnet en entrepriseforsikring. Klager medhold.}

\begin{center}\rule{0.5\linewidth}{\linethickness}\end{center}

Forsikringen dækker normalt ikke, hvis man har handlet forsætligt (med vilje). Det ses eksempelvis i Forsikringsankenævnets sag 86464:

\begin{center}\rule{0.5\linewidth}{\linethickness}\end{center}

\emph{Sag 86464: Klager over dækning for erstatningsansvar som følge af brand i solcenter. Klager og en veninde havde i et solcenter sat ild til papirstykker i kabinerne, som fik fat i en papirdispenser, der begyndte at brænde. Pigerne forlod solcentret og den rygende papirdispenser. Der var efterfølgende sket brandskader for 2,3 mio. kr. Klager blev idømt solidarisk ansvar med veninden for skaderne. Selskabet afviste at yde dækning med henvisning til at skaderne var sket med forsæt. Klager gjorde gældende, at det ikke var hende, der konkret havde sat ild til det papirstykke, der førte til ild i papirdispenseren og de efterfølgende brandskader. Nævnet fandt, at skaderne var sket med forsæt fra klager og lagde navnlig vægt på, at et solidarisk ansvar var omfattet af dækningsundtagelsen, og at klager ved sin medvirken og ved at forlade stedet med en rygende papirdispenser, der havde været ild i måtte have indset, at der herved var nærliggende risiko for yderligere skader. Selskab medhold.}

\begin{center}\rule{0.5\linewidth}{\linethickness}\end{center}

Har forsikringstageren handlet forsætligt, må han eller hun selv betale erstatning til den skadelidte. Har et barn under 14 år handlet forsætligt, dækker ansvarsforsikringen dog som udgangspunkt også sådanne forsætlige skader. Det samme er tilfældet, hvis man -- da man forvoldte skaden -- på grund af ens sindstilstand manglede evnen til at handle fornuftsmæssigt.

Man kan efter omstændighederne pålægges erstatningsansvar, hvis andre personer lider tab som følge af redningsaktioner til fordel for en selv. Privatansvarsforsikringen dækker også et sådant ansvar, i det omfang kravet ikke dækkes af lov om arbejdsskadeforsikring, og i det omfang redningsaktionen ikke sker som et led i redningsmandens erhverv.

Man bør som forsikringstager være tilbageholdende med at anerkende et erstatningsansvar på egen hånd, da man risikerer, at forsikringsselskabet er uenigt og derfor ikke vil dække skaden. Man bør derfor normalt kontakte sit forsikringsselskab og bede forsikringsselskabet om at tage stilling til, om man har pådraget sig et ansvar efter dansk rets almindelige erstatningsregler, som forsikringsselskabet vil dække i henhold til policen og forsikringsbetingelserne.

Bliver man sagsøgt af en person, der mener, at man er erstatningssvarlig som privatperson, kan det være hensigtsmæssigt at få en advokat til at føre ens sag. Hvis forsikringsselskabet er enigt i, at man ikke er erstatningsansvarlig, dækker forsikringen normalt også sådanne sagsomkostninger til en advokat.

Mener man, at forsikringsselskabets afvisning af at dække en skade -- som man har forvoldt på andres ejendom eller på andre personer -- er forkert, er det er en god ide at drøfte med sagsbehandleren i forsikringsselskabet, hvad der skal til, for at man kan bevise, at man har ret. Er man stadig utilfreds, kan man kontakte forsikringsselskabets klageansvarlige, hvis man mener, at sagsbehandleren, er nået til et forkert resultat i sagen.

Man skal altid have klaget til forsikringsselskabet over dets afgørelse, før man har adgang til at klage til Forsikringsankenævnet.

Det er værd at bemærke, at Forsikringsankenævnet ikke kan behandle en klage, hvis man som skadelidt mener, at skadevolderens forsikringsselskab skal erstatte de skader/tab, som skadevolderen har påført en som skadelidt. Det skyldes, at ankenævnet kun kan behandle klager over forsikringer, som man selv eller en i ens husstand har købt. Og da det ikke er skadelidtes forsikring, der rettes krav imod, kan Forsikringsankenævnet ikke behandle sagen.

Hvis man som skadelidt ikke kan blive enig med skadevolderen eller med vedkommendes forsikringsselskab, er skadelidte nødt til at anlægge en retssag ved domstolene mod skadevolderen. Skadelidte kan eventuelt få retshjælpsdækning til at føre en sådan retssag. Se f.eks. nævnte ankenævnssag:

\begin{center}\rule{0.5\linewidth}{\linethickness}\end{center}

\emph{FED 2013.16: Skadelidtes krav mod skadevolders ansvarsforsikring omfattede ikke policemæssig selvrisiko.}

\begin{center}\rule{0.5\linewidth}{\linethickness}\end{center}

Ofte kan der være oplysninger i en sag om erstatningsansvar, som ikke giver et entydigt billede af, hvad der er sket. Hvis der er tvivl om, hvad der er sket, vil forsikringsankenævnet altid vil foretage en konkret vurdering af validiteten af de oplysninger, som klageren (forsikringstageren) og forsikringsselskabet er fremkommet med. Forsikringsankenævnet vil normalt lægge størst vægt på de oplysninger, der er fremkommet fra uvildige parter. Og ankenævnet vil ofte lægge betydelig vægt på de forklaringer, som forsikringstageren eller en anden person først har givet til forsikringsselskabet eller til en tredjemand.

\hypertarget{uagtsom-forstlig-og-hndelig-skade-samt-gstebudsskade}{%
\subsection{Uagtsom, forsætlig og hændelig skade samt gæstebudsskade}\label{uagtsom-forstlig-og-hndelig-skade-samt-gstebudsskade}}

Privatansvarsforsikringen dækker som nævnt, hvis forsikringstageren juridisk set har pådraget sig et erstatningsansvar. For at pådrage sig et erstatningsansvar skal man bl.a. have udvist en uforsvarlig adfærd (handling eller undladelse), som kan tilregnes én som forsætlig eller uagtsom, og den skade, man har forvoldt, skal være en påregnelig følge af ens adfærd. Denne såkaldte culparegel står ikke skrevet i nogen lov, men er udviklet gennem langvarig retspraksis.

\hypertarget{uagtsom}{%
\subsubsection{Uagtsom}\label{uagtsom}}

Man har handlet uagtsomt, hvis en almindelig, fornuftig person (bonus pater) ved almindelig agtpågivenhed ville have indset, at skaden kunne indtræde som følge af ens adfærd. Ens adfærd vil med andre ord blive betragtet som uagtsom, hvis den afviger fra de i samfundet almindeligt anerkendte, forsvarlige handlemønstre. Det ses eksempelvis -- som allerede nævnt -- i Forsikringsankenævnets sag 86059, hvor ankenævnet fandt, at klageren havde handlet ansvarspådragende ved at lade en veninde til sin datter benytte en trampolin, hvis sikkerhedsnet ikke var monteret korrekt. Ligeledes er Forsikringsankenævnets sag 88426 om en 12-årig dreng, der skubber en voksen i ryggen, samt sagerne 87739, 86992 og 85472 relevante.

\hypertarget{forstlig}{%
\subsubsection{Forsætlig}\label{forstlig}}

Man har handlet forsætligt, hvis man havde vilje til at forårsage den skade, der indtrådte, hvis man havde indset, at skaden ville indtræde, eller hvis man ikke ville have handlet anderledes, selv om man havde indset, at skaden med stor sandsynlighed ville indtræde. Det ses eksempelvis i tidligere nævnte Forsikringsankenævns sag 86464, hvor nævnet fandt, at brandskaderne på et solcenter var sket med forsæt, da skadevolderen måtte have indset, at der var nærliggende risiko for skade, idet en papirdispenser røg, da skadevolderen forlod solcentret.

Privatansvarsforsikringen dækker ikke, hvis man har forvoldt skaden forsætligt. Har man handlet forsætligt, bliver man selvfølgelig erstatningsansvarlig over for skadelidte, men det er ikke noget, som ansvarsforsikringen dækker. Her må skadevolder selv betale. Dette gælder dog ikke for børn under 14 år, der handler forsætligt, medmindre det klart fremgår af forsikringsbetingelserne, at børns forsætlige handlinger ikke dækkes. Det følger af forsikringsaftalelovens (Lovbekendtgørelse nr. 1237 af 9. november 2015) § 19, stk. 1:

Bestemmelserne i § 18 om bortfald eller begrænsning af selskabets ansvar finder ikke anvendelse, når den sikrede var under 14 år eller på grund af sindssygdom, åndssvaghed, forbigående sindsforvirring eller lignende tilstand har manglet evnen til at handle fornuftsmæssigt.

Se fra retspraksis:

\begin{center}\rule{0.5\linewidth}{\linethickness}\end{center}

\emph{U 1964.786 H: To 7-årige drenge havde begået hærværk på en bigård og herved pådraget sig erstatningsansvar. Deres forældre havde tegnet familieansvarsforsikringer. Da de i de pågældende selskabers policer indeholdte vilkår ikke med tilstrækkelig klarhed udelukkede anvendelse af den særlige bestemmelse i forsikringsaftalelovens. § 19, stk. 1, jfr. § 18, tilpligtedes selskaberne at friholde drengene for erstatningsansvaret.}

\emph{U 1954 746 H: Hvor en mand fik udbetalt en ulykkeserstatning for hustruen uanset, at han havde dræbt hende og deres 6 børn. Da han var sindssyg i gerningsøjeblikket, lidende af en psykogen sindssygdom, måtte det antages, at han ved begåelsen af drabet på grund af sindssygdom havde manglet evnen til at handle fornuftsmæssigt og da drabet måtte anses for en af forsikringen omfattet begivenhed, fandtes han i medfør af forsikringsaftalelovens § 19, stk. 1, at have krav på forsikringsydelsen.}

\emph{U 1982 528 H: Hvor en mand dræbte sin hustru og hund, hvorefter han satte ild på sit hus, hvorved han indebrændte. Det var ubestridt, at han havde handlet forsætligt, jf. forsikringsaftalelovens § 18, stk. 1. Retslægerådet mente, at der ikke var sikre holdepunkter for, at mandens handling skyldtes sindssygdom eller lignende tilstand. Retten antog -- på grund af det abnorme handlingsforløb -- at manden havde befundet sig i en sygelig sindstilstand, hvorfor forsikringsselskabet skulle udbetale forsikringssummen.}

\emph{U 2006 1961 Ø: Hvor en mand satte i suicidal øjemed ild til sin ejendom. Retslægerådet skønnede at han ikke var sindssyg og heller ikke var det på gerningstidspunktet. Rådet fandt dog -- med henvisning til omfattende alkoholmisbrug -- at manden mest sandsynlig var omfattet af straffelovens § 69. Landsretten bemærkede, at forsikringsaftalelovens § 19, stk. 1, har et bredere anvendelsesområde end straffelovens § 16, idet forsikringsaftalelovens § 19, stk. 1, blandt andet omfatter »forbigående sindsforvirring eller lignende tilstand«. Bevisbyrden for, at den sikrede ved fremkaldelse af forsikringsbegivenheden opfylder det psykiske kriterium, ligger hos kravstilleren, jf. U 1982 528 H, nævnt ovenfor. Det er en relativ tung bevisbyrde.}

\begin{center}\rule{0.5\linewidth}{\linethickness}\end{center}

Et eksempel på en forsætlig skade er Forsikringsankenævnets sag 86048, hvor nævnet fandt, at klagerens søn havde handlet forsætligt, da han -- mens han stod med ryggen til og i irritation over at være blevet låst ude af nogle klassekammerater -- sparkede bagud mod en dør og en siderude. Nævnet fandt, at det var overvejende sandsynligt, at ruden ville gå i stykker ved en sådan handling.

\hypertarget{hndelig}{%
\subsubsection{Hændelig}\label{hndelig}}

Forvolder man en skade hændeligt, dækker ens ansvarsforsikring heller ikke. At det er hændeligt, betyder, at skaden indtræffer tilfældigt, og uden at det skyldes uagtsomhed eller forsømmelse fra ens side. I den situation bliver man ikke juridisk erstatningsansvarlig efter dansk ret. Derfor må skadelidte i en sådan situation selv bære tabet, med mindre skadevolder alligevel betaler, fordi skadevolder føler, at skadevolder er moralsk ansvarlig. Men dette er ansvarsforsikringen uvedkommende. Af relevante sager kan nævnes Forsikringssankenævnets sag 86248, hvor en kvinde tissede i en seng, samt forsikringsankenævnets sag 81840 om en kajak, der i en carport faldt ned på en bil:

\begin{center}\rule{0.5\linewidth}{\linethickness}\end{center}

\emph{Sag 86248: Klager over afvisning af dækning for en skade på klagers vens seng forårsaget af klagers ufrivillige vandladning. Selskabet henviste til, at uheldet var hændeligt, og at klager ikke havde handlet uagtsomt, hvorfor selskabet ikke kunne anerkende, at hun havde handlet ansvarspådragende. Nævnet fandt, at klager ikke havde udvist fejl eller forsømmelser forud for skadens indtræden, hvorfor klager ikke kunne pålægges et ansvar for den indtrufne skade. Nævnet fandt endvidere, at klagers lejlighedsvise brug af et præparat mod ufrivillig vandladning ikke kunne ændre herpå. Selskab medhold.}

\emph{Sag 81840: Klager over afvisning af at dækning af ansvarsskade. Klagers kajak, der var opmagasineret i hans tidligere udlejers carport, blæste under storm ned fra sin placering mellem to bjælker og forårsagede herved skade på udlejers bil. Selskabet afviste dækning med henvisning til, at skaden var hændelig. Selskab medhold.}

\begin{center}\rule{0.5\linewidth}{\linethickness}\end{center}

\hypertarget{gstebudsskade}{%
\subsubsection{Gæstebudsskade}\label{gstebudsskade}}

Forvolder man skaden under almindeligt privat samvær -- typisk af selskabelig karakter -- dækker ansvarsforsikringen i en række situationer, hvor domstolene vil være tilbageholdende med at pålægge en et juridisk erstatningsansvar. Disse skader kaldes normalt for ``gæstebudsskader''. Et eksempel er Forsikringsankenævnets sag 75029, hvor klagerens datter -- mens hun var på besøg -- beskadigede et trægulv med sine spidse stilethæle. Der kan også henvises til Forsikringsankenævnets sag 75365, samt sammenstød af cykler under en fælles motionsaktivitet, jf. forsikringsankenævnets sag 91226:.

\begin{center}\rule{0.5\linewidth}{\linethickness}\end{center}

\emph{Sag 91226: Klager over afvisning af dækning for skade ved sammenstød af cykler. Klager og skadelidte havde deltaget i en fælles cykeltur, hvor klager var stødt ind i baghjulet på skadelidtes cykel. Klager ønskede, at selskabet skulle erstatte skade på baghjulet med henvisning til, at han havde handlet ansvarspådragende ved at se bagud under kørslen, alternativt at selskabet dækkede skaden som gæstebudsskade. Selskabet anførte, at klagers erstatningsansvar skulle vurderes på baggrund af den milde ansvarsvurdering, der skete ved skader forvoldt idrætsudøvere imellem under udøvelse af sportsaktiviteter. Selskabet anførte yderligere, at gæstebudsdækningen ikke var relevant, da formålet med cykelturen havde været motion og ikke den personlige relation eller samværet. Nævnet fandt, at klageren ikke havde handlet ansvarspådragende, hvorfor selskabet ikke skulle dække efter den almindelige regel om erstatningsansvar. Nævnet fandt, at der måtte sondres mellem selve motionscyklingen - som ikke ville være omfattet af reglen om gæstebudsskade - og det efterfølgende sociale samvær, hvor man ifølge klageren ``får en kop kaffe og en øl til en snak efterfølgende'', idet denne del af samværet ville være omfattet af den særlige dækning, som forsikringen yder ved gæstebudsskader, idet dette samvær falder ind under begrebet ``almindeligt privat samvær''. Klageren havde derfor ikke krav på, at skaden på cyklen blev dækket efter reglen om gæstebudsskade. Selskab medhold.}

\begin{center}\rule{0.5\linewidth}{\linethickness}\end{center}

\begin{longtable}[]{@{}ccc@{}}
\toprule
\begin{minipage}[b]{0.19\columnwidth}\centering
Hvordan er handlingen forvoldt?\strut
\end{minipage} & \begin{minipage}[b]{0.26\columnwidth}\centering
Er der juridisk set erstatningsansvar?\strut
\end{minipage} & \begin{minipage}[b]{0.46\columnwidth}\centering
Dækker ansvarsforsikringen?\strut
\end{minipage}\tabularnewline
\midrule
\endhead
\begin{minipage}[t]{0.19\columnwidth}\centering
Forsætligt\strut
\end{minipage} & \begin{minipage}[t]{0.26\columnwidth}\centering
Ja\strut
\end{minipage} & \begin{minipage}[t]{0.46\columnwidth}\centering
Nej\footnotemark{}\strut
\end{minipage}
\footnotetext{Hvis skadevolder er under 14 år eller er utilregnelig, kan der dog være dækning.}\tabularnewline
\begin{minipage}[t]{0.19\columnwidth}\centering
Uagtsomt\strut
\end{minipage} & \begin{minipage}[t]{0.26\columnwidth}\centering
Ja\strut
\end{minipage} & \begin{minipage}[t]{0.46\columnwidth}\centering
Ja\strut
\end{minipage}\tabularnewline
\begin{minipage}[t]{0.19\columnwidth}\centering
Gæstebudsskade\strut
\end{minipage} & \begin{minipage}[t]{0.26\columnwidth}\centering
Ja\strut
\end{minipage} & \begin{minipage}[t]{0.46\columnwidth}\centering
Ja\strut
\end{minipage}\tabularnewline
\begin{minipage}[t]{0.19\columnwidth}\centering
Hændeligt\strut
\end{minipage} & \begin{minipage}[t]{0.26\columnwidth}\centering
Nej\strut
\end{minipage} & \begin{minipage}[t]{0.46\columnwidth}\centering
Nej\strut
\end{minipage}\tabularnewline
\bottomrule
\end{longtable}

Kilde: Forsikringsankenævnets klageguide

Er skaden dækket af en tingsforsikring tegnet af skadelidte, f.eks. en indboforsikring eller en kaskoforsikring på hus eller bil, er skadevolder som udgangspunkt ikke erstatningsansvarlig for skaden, og skadevolders ansvarsforsikring skal derfor ikke dække. Skadelidte må i stedet rette henvendelse til sit eget forsikringsselskab.

Der gælder visse undtagelser fra dette princip. Eksempelvis er der alligevel erstatningsansvar -- og dækning hos ens ansvarsforsikringsselskab -- hvis man har forvoldt skaden ved grov uagtsomhed. Ved grov uagtsomhed forstås normalt, at den udviste adfærd indebar en indlysende fare for den skade, som faktisk skete. Der er ligeledes erstatningsansvar, hvis man har forvoldt skaden forsætligt, men i de tilfælde dækker ens ansvarsforsikring som ovenfor anført ikke.

\hypertarget{ansvarsforsikringens-undtagelser}{%
\subsection{Ansvarsforsikringens undtagelser}\label{ansvarsforsikringens-undtagelser}}

Privatansvarsforsikringen indeholder en række undtagelser, hvor ansvarsforsikringen ikke dækker. De mest almindelige undtagelser er ifølge Forsikringsankenævnet følgende:

Ansvar for skade, der rammer den sikrede personkreds
Ansvarsforsikringen dækker normalt ikke, hvis man kommer til at forvolde skade på en af de personer, som familie-/indboforsikringen omfatter. Det vil typisk være tilfældet, hvis man forvolder skade på ejendele, der tilhører en selv, da man ikke kan være erstatningsansvarlig over for sig selv, ens ægtefælle/samlevende eller ens børn. Dog dækker ansvarsforsikringen normalt, hvis man forvolder personskade på dem.

Ansvarsforsikringen dækker som regel kun ansvar, man pådrager sig i privatlivet
Forvolder man skade i forbindelse med sit erhverv, skal dækning søges på en eventuel erhvervsansvarsforsikring. Hvis det erhvervsmæssige aspekt i aktiviteten er helt underordnet -- f.eks. hvis man kommer til at ødelægge en stol, mens man er til et bestyrelsesmøde i andelsboligforeningen, der afholdes hos en anden andelshaver -- vil ens ansvarsforsikring normalt dække.

Ansvar for arbejde for andre
Hvis man arbejder for andre og får løn for det, så dækker privatansvarsforsikringen ikke de skader, man forvolder som ansat.

Hvis man ikke får løn, dækker ansvarsforsikringen i nogle tilfælde. Man ser på, om man selv har en interesse i det arbejde, som man udfører, eller om det er en vennetjeneste. Herudover ser man på omfanget og intensiteten af ens arbejdsindsats. Hvis man kortvarigt hjælper ens nabo med f.eks. at støvsuge og taber støvsugerrøret ned på et glasbord, der smadres, vil ansvarsforsikringen normalt dække. Lignende sager ses i Forsikringsankenævnet i sagerne 39930 og 60013:

\begin{center}\rule{0.5\linewidth}{\linethickness}\end{center}

\emph{Sag 39930: Ansvar -- ansvar. Klagers søn forvolder skade på sin brors linoliegulv, da han forsøger at fjerne tæpperester med ovnrens -- anset for arbejde for andre.}

\emph{Sag 60013: Ansvar -- vennetjeneste. Selskabet afviser dækning for ansvarsskade. Klageren ville være sød og støvsuge et værelse hos udlejer. Smadrede glasbord. Selskabet gør gældende, at klageren ikke er erstatningsansvarlig for den skete skade, der er sket ved simpel uagtsomhed under udøvelse af en tjeneste i skadelidtes interesse. Nævnet finder, at klageren, der ikke kan henføres under undtagelsesbestemmelsen om arbejde for andre, har handlet uagtsomt. Klageren er derved erstatningsansvarlig for den forvoldte skade. Klageren medhold.}

\begin{center}\rule{0.5\linewidth}{\linethickness}\end{center}

Ansvar i kontraktforhold
Hvis man bliver erstatningsansvarlig, fordi man ikke overholder en aftale eller en kontrakt, dækker ansvarsforsikringen ikke.

Ansvar for ting, der er i ens varetægt
Ofte dækker ansvarsforsikringen ikke ansvar for skade på ting, som man har til låns, leje eller opbevaring, afbenyttelse, befordring, eller som af andre grunde befinder sig i ens varetægt. Låner man eksempelvis en iPad af en ven i længere tid, vil iPaden være i ens varetægt, og ansvarsforsikringen dækker ikke.

Her skal det bemærkes, at nogle ansvarsforsikringer dækker eksempelvis i den første måned, hvor man låner eller lejer en genstand.

Ens råden over genstanden skal være af en vis intensitet, for at man kan siges at have den i sin varetægt. Cykler man eksempelvis over til en ven med hans iPad, som han har glemt hjemme hos en, og taber man den på vejen, vil der typisk være forsikringsdækning. Ved kortvarig råden over en genstand med ejerens accept vil der som udgangspunkt også være dækning på ansvarsforsikringen.

Forsikringsankenævnet har i sagen 85614 taget stilling til, om undervisningsmateriale udleveret af en skole -- herunder en iPad -- er omfattet af den private ansvarsforsikring. Nævnet fandt, at dette ikke var tilfældet, da lånet af undervisningsmaterialerne ikke udelukkende var i elevens interesse:

\begin{center}\rule{0.5\linewidth}{\linethickness}\end{center}

\emph{Sag 85614: Klager over afslag dækning for en iPad og skolebøger. Genstandene lå i klagers skoletaske, der blev stjålet. Klager anførte, at de var udlånt af gymnasiet i hendes interesse til brug for uddannelsen, hvorfor hun bar risikoen for dem. Hun henviste til både DL 5-8-1 (Danske Lov fra 1683) og en underskrevet kontrakt med skolen om lån af iPad. Selskabet henviste til, at genstandene tilhørte skolen, samt til udtalelser fra Folketingets Ombudsmand og Undervisningsministeriet, hvoraf fremgik, at DL 5-8-1 ikke fandt anvendelse ved skolers udlån af undervisningsmidler, og at der ikke gyldigt kunne indgås aftale om et mere vidtrækkende ansvar for de lånte genstande, end hvad der fulgte af den almindelige erstatningsretlige culparegel. Nævnet fandt, at klager ikke ejede genstandene og heller ikke bar risikoen for disse. Nævnet måtte lægge til grund, at kontraktens vilkår om, at klager bar tyveririsikoen for iPaden var ugyldigt, da vilkåret gik videre end dansk rets almindelige regler om erstatning. Nævnet fandt, at ansvar for skader på lånte genstande efter betingelserne var undtaget dækning, når der var gået mere end 30 dage efter overtagelsen. Selskabet var derfor ikke forpligtet til at dække tyveriet. Selskab medhold.}

\begin{center}\rule{0.5\linewidth}{\linethickness}\end{center}

Skader som f.eks. tyveri, brand og lignende, der kan overgå genstande, man har lånt, lejet, eller som på anden måde er i ens varetægt, vil være omfattet af ens indboforsikring.

Det samme gælder genstande, som børn kommer til at ødelægge, fordi de har en naturlig opdagetrang. Et vidtgående eksempel er U 1980.1082 Ø. Her skulle forsikringsselskabet erstatte, at to 16-årige drenge havde formået at starte en bulldozer med en skruetrækker, hvorefter bulldozeren kørte i havnen:

\begin{center}\rule{0.5\linewidth}{\linethickness}\end{center}

\emph{U 1980.1082 Ø: To 16-årige drenge A og B brugsstjal en bulldozer på et havneområde. Kort efter at A ved hjælp af en skruetrækker, som B rakte ham, havde startet bulldozeren, kørte den i havnen. hvorved der opstod betydelig skade på bulldozeren. B's fader havde i et forsikringsselskab tegnet en privatforsikring. Undtaget fra selskabets ansvar var ``forsætlige skader'', ``skader på ting, der var i den sikredes varetægt'' og ``skader forvoldt under benyttelse af motordrevet køretøj''. B fandtes erstatningsansvarlig efter almindelige regler, da han ansås for forsætligt at have medvirket til, at bulldozeren begyndte at køre, og da han og A ved uagtsomt forhold havde forvoldt skaden, der måtte anses for påregnelig. Skaden ansås dækket af forsikringspolicen, da skaden ikke var forårsaget ved forsæt, da et varetægtsforhold ikke ansås for opstået ved den kortvarige benyttelse af bulldozeren, og da undtagelsesbestemmelsen vedrørende motordrevet køretøj efter sit indhold ikke klart omfattede skaden på det benyttede køretøj. Erstatningen omfattede skadelidtes ``egne leverancer'' og et beløb til dækning af administrationsudgifter.}

\begin{center}\rule{0.5\linewidth}{\linethickness}\end{center}

Ansvar forvoldt ved brug af motorkøretøjer, ens ansvar som husejer eller som hundeejer
Sådanne skader dækkes ikke, og man skal således købe en autoansvarsforsikring, en husejerforsikring med ansvarsdækning eller en hundeforsikring.

\hypertarget{quiz-1}{%
\section{Quiz}\label{quiz-1}}

Quiz Erstatning

\hypertarget{kb}{%
\chapter{Køb}\label{kb}}

\hypertarget{quiz-2}{%
\section{Quiz}\label{quiz-2}}

Quiz Køb

\hypertarget{markedsfring}{%
\chapter{Markedsføring}\label{markedsfring}}

\hypertarget{persondatabeskyttelse-og-hvidvask}{%
\chapter{Persondatabeskyttelse og hvidvask}\label{persondatabeskyttelse-og-hvidvask}}

\hypertarget{virksomhedsformer-og-hftelser}{%
\chapter{Virksomhedsformer og hæftelser}\label{virksomhedsformer-og-hftelser}}

\hypertarget{insolvensret}{%
\chapter{Insolvensret}\label{insolvensret}}

\hypertarget{kreditaftaler}{%
\chapter{Kreditaftaler}\label{kreditaftaler}}

\hypertarget{kautionsforhold}{%
\chapter{Kautionsforhold}\label{kautionsforhold}}

\hypertarget{fordringer-gldsbreve-og-pantebreve}{%
\chapter{Fordringer gældsbreve og pantebreve}\label{fordringer-gldsbreve-og-pantebreve}}

\hypertarget{overdragelse-af-fordringer}{%
\chapter{Overdragelse af fordringer}\label{overdragelse-af-fordringer}}

\hypertarget{pant-og-tinglysning}{%
\chapter{Pant og tinglysning}\label{pant-og-tinglysning}}

\hypertarget{tingsretlige-konflikter}{%
\chapter{Tingsretlige konflikter}\label{tingsretlige-konflikter}}

\hypertarget{handel-med-fast-ejendom}{%
\chapter{Handel med fast ejendom}\label{handel-med-fast-ejendom}}

\hypertarget{radgiveransvar-og-god-skik}{%
\chapter{Rådgiveransvar og god skik}\label{radgiveransvar-og-god-skik}}

\hypertarget{familie-og-arveret}{%
\chapter{Familie og arveret}\label{familie-og-arveret}}

\hypertarget{forsikring}{%
\chapter{Forsikring}\label{forsikring}}

\hypertarget{lejeret}{%
\chapter{Lejeret}\label{lejeret}}

\hypertarget{de-finansielle-tvistlsningsorganer}{%
\chapter{De finansielle tvistløsningsorganer}\label{de-finansielle-tvistlsningsorganer}}

\hypertarget{lovsamling}{%
\chapter{Lovsamling}\label{lovsamling}}

Aftaleloven

Arveloven

Betalingsloven

Checkloven

Datasammenskrivning af Kong Christian den femtis danske lov

E-handelsloven

Erstatningsansvarsloven

Forbrugeraftaleloven

Forbrugerbeskyttelse ved erhvervelse af fast ejendom mv

Forsikringsaftaleloven

Grundloven

Gældsbrevsloven

Konkursloven

Kreditaftaleloven

Købeloven

Lejeloven

Lov om andelsboligforeninger og andre bofællesskaber

Lov om finansiel virksomhed

Lov om formidling af fast ejendom

Lov om ægtefællers økonomiske forhold

Markedsføringsloven

Selskabsloven

Tinglysningsloven

Værgemålsloven

\bibliography{book.bib,packages.bib}


\end{document}
